\chapter{Was ist Algebra?}
\label{Was ist Algebra}

Die Algebra ist eines der grundlegendsten Teilgebiete der Mathematik, das sich mit dem Lösen von Gleichungen, Rechenoperationen und allgemein mit mathematischen Strukturen hinter diesen Operationen befasst. In diesem Artikel werden wir dieses Teilgebiet kurz vorstellen.

\section{Geschichtliches}
\subsection{Zeit der Griechen}
Bereits die alten Griechen untersuchten algebraische Gleichungen<ref>vgl. auch den Wikipedia-Artikel [[w:Algebra|„Algebra“]]</ref>. Jedoch waren sie – anders als die Babylonier und Ägypter – nicht nur an der Lösung praktischer Probleme interessiert. Sie sahen insbesondere geometrische Fragestellungen als zentrales Teilgebiet ihrer Philosophie an. Dies war der Beginn der Algebra, der Geometrie und der Mathematik als Wissenschaft insgesamt.

Bei den Griechen stellten Seiten (meist Strecken) geometrischer Objekte  die Terme algebraischer Gleichungen dar. Mittels Konstruktionsverfahren mit Zirkel und Lineal bestimmten sie die dazugehörigen Lösungen. Da die altgriechische Algebra durch die Geometrie begründet wurde, spricht man von der geometrischen Algebra.

Einer der bedeutendsten griechischen Mathematiker war Euklid von Alexandria. Der zweite Band der von ihm verfassten ''Elemente'' enthält eine Reihe von algebraischen Aussagen, die in der Sprache der Geometrie formuliert wurden. Euklid diskutierte in den Elementen unter anderem die Theorie der Flächenanlegung, die auf die Altpythagoreer zurückgeht. Mit dieser Methode kann man aus Sicht der modernen Algebra bestimmte lineare und quadratische Gleichungen mit einer Unbestimmten lösen.

Im zehnten Buch der ''Elemente'' lieferte Euklid einen Beweis der Irrationalität von $\sqrt 2$. Irrationale Größenverhältnisse waren bereits den Pythagoreern bekannt. Diese hatten auch obigen Satz von Euklid in allgemeinerer Form bewiesen.

Diophantos von Alexandria, der wahrscheinlich um das Jahr 250 n. Chr. lebte, gilt als der bedeutendste Algebraiker der Antike. Sein erstes und wichtigstes Werk, die ''Arithmetica'', bestand ursprünglich aus dreizehn einzelnen Büchern, von denen aber nur sechs überliefert sind. Mit diesem Werk löste er die damals bekannte Arithmetik und Algebra vollständig von der Geometrie ab.

\subsection{Geometrie und Algebra}
[[File:Frans Hals - Portret van René Descartes.jpg|thumb|René Descartes]]
[[File:Cartesian-coordinate-system.svg|thumb|Das kartesische Koordinatensytem mit vier eingezeichneten Punkten]]
Die Verbindung von Geometrie und Algebra zur sogenannten analytischen Geometrie war einer der größten Fortschritte in der Mathematik. Sie ermöglicht die rechnerische Lösung vieler geometrischer Probleme<ref name="Stillwell">Stillwell, John (2010). „Mathematics and its History“ (3. Auflage). Springer Science + Business Media Inc. p. 109. ISBN 144196052X. „Around 1630, both Fermat and Descartes realized that geometric problems could be translated into algebra by means of ''coordinates'', and that many problems could then be routinely solved by algebraic manipulation.“</ref>. Ausgangspunkt ist das kartesische Koordinatensystem,  welches nach [[w:René Descartes|René Descartes]] benannt ist<ref>Wikipedia-Artikel [https://de.wikipedia.org/w/index.php?title=Kartesisches Koordinatensystem\&oldid=148345057 „Kartesisches Koordinatensystem“]. Abgerufen am 28. Januar 2016. „Das kartesische Koordinatensystem ist nach dem latinisierten Namen Cartesius des französischen Mathematikers René Descartes benannt, der das Konzept der „kartesischen Koordinaten“ bekannt gemacht hat.“</ref>. René Descartes (1596 – 1650) war neben [[w:Pierre de Fermat|Pierre de Fermat]] (1607 – 1665) einer der Begründer der analytischen Geometrie<ref name="Stillwell"/>.

Im kartesischen Koordinatensystem werden Punkte in der Ebene beziehungsweise im Raum durch Zahlen ausgedrückt, was eine algebraische Beschreibung von geometrischen Figuren ermöglicht. So wird durch die Gleichung $x^2+y^2=4$ ein Kreis um den Nullpunkt mit dem Radius $2$ beschrieben. Die Gleichung $y=2x$ stellt eine Gerade durch den Nullpunkt mit Steigung $2$ dar.

[[File:Cartesian coordinate system with circle and line.svg|zentriert|350px|Der Kreis um den Nullpunkt mit Radius 2 und die Gerade durch den Nullpunkt mit Steigung 2]]

Diese algebraische Beschreibung ermöglicht durch gewisse Berechnungen die Lösung von geometrischen Problemen. Wenn du beispielsweise die Schnittpunkte des Kreises $x^2+y^2=4$ mit der Geraden $y=2x$ berechnen möchtest, so musst du die Lösungen des folgenden Gleichungssystems bestimmen:

\begin{align*}
x^2+y^2& =4 \\
y & = 2x \\
\end{align*}

\textbf{Zusammenfassung:} Die analytische Geometrie erfasst und untersucht geometrische Gebilde mit algebraischen Hilfsmitteln, wobei die Einführung des Koordinatensystems der Ausgangspunkt dieser neuen Theorie war. Bis heute ist der Einsatz des Computers in der Geometrie nur dank der analytischen Geometrie möglich.


\begin{verstaendnisfrage}{Welche Koordinaten haben die Schnittpunkte des Kreises $x^2+y^2=4$ mit der Geraden $y=2x$?}
Zunächst setzt man die zweite Gleichung in die erste ein und erhält:

\begin{align*}
&& x^2+y^2& =4 \\
&&& {\color{green} \left\downarrow\ y=2x \right.} \\
&\implies & x^2 + 4x^2 &= 4 \\
&\implies & 5x^2 &= 4 \\
&\implies & x^2 &= \frac{4}{5}
\end{align*}

Diese Gleichung hat zwei Lösungen: $x_1 = \tfrac{2}{\sqrt {5}}$ und $x_2 = -\tfrac{2}{\sqrt {5}}$. Die zu diesen beiden Lösungen dazugehörigen $y$-Werte sind

\begin{align*}
y_1 &= 2 x_1 = 2\cdot \frac{2}{\sqrt {5}} = \frac{4}{\sqrt {5}} \\[0.5em]
y_2 &= 2 x_2 = 2\cdot \left(-\frac{2}{\sqrt {5}}\right) = -\frac{4}{\sqrt {5}}
\end{align*}

Damit sind die beiden Schnittpunkte der Geraden $y=2x$ mit dem Kreis $x^2+y^2=4$ die beiden Punkte $S_1 = \left( \tfrac {2}{\sqrt 5} ; \tfrac {4}{\sqrt 5}  \right)$ und $S_2 = \left( - \tfrac{2}{\sqrt 5} ;  - \tfrac {4} {\sqrt 5} \right)$.
\end{verstaendnisfrage}

\subsection{Die Entwicklung des Vektorraumbegriffs}
[[File:Hermann Graßmann.jpg|thumb|Hermann Graßmann]]
Die Behandlung von Koordinaten führte zum Begriff des Vektorraums. Dieser wurde erst im 19. Jahrhundert von [[w:Hermann Graßmann|Hermann Graßmann]] (1809  – 1877) in seinem mathematisch größten Werk „Ausdehnungslehre“ eingeführt<ref name=":0">Siehe den Wikipedia-Artikel [[w:Hermann Graßmann|„Hermann Graßmann“]]</ref>.  Hier beschrieb er als Erster eine $n$-dimensionale euklidische Geometrie mit Hilfe von Vektoren. Damit gilt er als Begründer der Vektorrechnung. 

Jedoch wurde erst spät die Bedeutung seiner wissenschaftlichen Arbeit erkannt<ref name=":0" /> und bis heute ist er als Mathematiker kaum bekannt. Die Theorie von Graßmann beschrieb viele Begriffe der heutigen linearen Algebra wie Dimension, Erzeugendensystem und Basen (nur unter anderen Namen)<ref name=":0" />. Giuseppe Peano (1858 – 1932) war der Erste, der aufbauend auf der Theorie von Graßmann die moderne Definition eines Vektorraums formulierte<ref>Peano, Giuseppe (1888), ''Calcolo Geometrico secondo l'Ausdehnungslehre di H. Grassmann preceduto dalle Operazioni della Logica Deduttiva'', Turin</ref>. 

Neben ihrer geometrischen Bedeutung, die in der graphischen Datenverarbeitung und in der Robotik besonders wichtig ist, ist die lineare Algebra als Theorie der Vektorräume in der gesamten Mathematik ein äußerst wichtiges Gebiet. Vektorräume werden auch in vielen Anwendungsbereichen der Mathematik, von den Wirtschaftswissenschaften bis zur Physik, benötigt.

\section{Teilbereiche der Algebra}
\begin{itemize}
\item '''Elementare Algebra:''' Die elementare Algebra ist die Algebra im Sinne der Schulmathematik. Sie umfasst die Rechenregeln der natürlichen, ganzen, gebrochenen und reellen Zahlen. Auch untersucht sie den Umgang mit Ausdrücken, die Variablen enthalten, und findet Wege zur Lösung einfacher algebraischer Probleme wie das Auflösen quadratischer Gleichungen.
\item '''Klassische Algebra:''' Die klassische Algebra beschäftigt sich mit dem Lösen allgemeiner algebraischer Gleichungen über den reellen oder komplexen Zahlen. Ihr zentrales Resultat ist der Fundamentalsatz der Algebra. Dieser besagt, dass jedes nicht konstante Polynom $n$-ten Grades in $n$ Linearfaktoren mit komplexen Koeffizienten zerlegt werden kann.
\item '''Lineare Algebra:''' Die lineare Algebra beschäftigt sich mit Vektorräumen und linearen Abbildungen zwischen diesen. Dabei sind Vektorräume Verallgemeinerungen der aus der Schule bekannten Vektorräume $\R^2$ und $\R^3$. Dies schließt insbesondere auch die Betrachtung von linearen Gleichungssystemen und Matrizen mit ein. Die Vektorräume $\R^n$ mit $n \in \N$ nennt man ''euklidische Vektorräume'' oder auch ''Koordinatenräume''. Lineare Algebra ist auch die Grundlage für die analytische Geometrie. Diese ist ein Teilgebiet der Geometrie, das algebraische Hilfsmittel (vor allem aus der linearen Algebra) zur Lösung geometrischer Probleme bereitstellt. Sie ermöglicht es in vielen Fällen, geometrische Aufgabenstellungen rein rechnerisch zu lösen, ohne die Anschauung zu Hilfe zu nehmen<ref>siehe dazu [[w:Analytische Geometrie|Analytische Geometrie]]</ref>.
\item '''Abstrakte Algebra:''' Die abstrakte Algebra ist die Grundlagendisziplin der modernen Mathematik. Sie beschäftigt sich mit algebraischen Strukturen wie Gruppen, Ringen, Körpern usw. Dabei untersucht sie ihre Eigenschaften und die Abbildungen, die es zwischen diesen Strukturen gibt. Die in der abstrakten Algebra untersuchten Strukturen tauchen in vielen Teilgebieten der Mathematik auf. So verbindet die abstrakte Algebra viele mathematische Theorien und findet viele Anwendungen.
\end{itemize}

\section{Was ist Algebra?}

[[File:Niels Henrik Abel.jpg|thumb|Niels Henrik Abel]]
[[File:Evariste Galois.jpg|thumb|Evariste Galois]]
[[File:Arthur Cayley.jpg|thumb|Arthur Cayley]]

Die Definition des mathematischen Teilgebiets ''Algebra'' ist nicht eindeutig und die jeweiligen Bereiche sind oft sehr unterschiedlich. 

Ursprünglich versteht man unter Algebra das Auflösen und Umformen von Gleichungen durch Rechnen mit Symbolen. Dies zeigt sich auch im Ursprung des Namens „Algebra“. Das arabische <al-ǧabr> bedeutet nämlich „Einrenken gebrochener Knochen“. In der Mathematik meint die Algebra gewissermaßen das „Einrenken von mathematischen Termen“.<ref Name="Definition">http://www.wiley-vch.de/books/sample/3527707212 c01.pdf, Seite 29</ref> Es geht also darum Lösungen für unbekannte Terme zu finden. Im Unterschied zur Analysis wird in der Algebra auf Grenzwertbildung verzichtet — Gleichungen sollen immer in endlich vielen Rechenschritten exakt gelöst werden und nicht zum Beispiel durch Reihenentwicklung oder Approximation. 

Wenn die Gleichungen linear sind, spricht man von linearer Algebra; durch Elimination lassen sich im Prinzip beliebig viele lineare Gleichungen mit beliebig vielen Variablen lösen. Sind die Gleichungen nicht linear, ist das nicht mehr so einfach möglich. Lineare und quadratische Gleichungen in einer Variablen wurden schon im Altertum (zuerst von den Babyloniern) untersucht. Heute lernt man das Lösen dieser Gleichungen in der Schule. Entsprechende allgemeine Formeln für Gleichungen dritten und vierten Grades wurden im 16. Jahrhundert in Italien gefunden. Die Suche nach Lösungsformeln für Gleichungen vom Grad 5 blieb lange erfolglos. Im Jahr 1824 bewies schließlich Abel, aufbauend auf einem unvollständigen Beweis von Ruffini, dass geschlossene Lösungsformeln nicht existieren<ref>siehe Wikipedia-Artikel [[w:Satz von Abel-Ruffini|„Satz von Abel-Ruffini“]]</ref>. Der Beweis dieser erstaunlichen Aussage ist ein Meilenstein in der Geschichte der Mathematik. In den darauf folgenden Jahrzehnten wurde die Theorie erheblich weiter entwickelt, unter anderem durch [[w:Évariste Galois|Galois]], wodurch auch die Prinzipien hinter dem Satz von Abel-Ruffini klarer wurden. Daraus entstand schließlich, unter dem Einfluss von [[w:Arthur Cayley|Arthur Cayley]] und anderen, die Theorie der Gruppen und damit allmählich die abstrakte Algebra als Strukturtheorie.

Die systematische Untersuchung von Strukturen ist ein wesentliches Merkmal der modernen Mathematik. Heute lernt man schon im ersten Semester, was Körper, Vektorräume und Gruppen sind und bekommt ein Gefühl für die abstrakte Herangehensweise an die Lösungstheorie von Gleichungen und andere Fragen vermittelt. Andererseits erscheint die reine Strukturtheorie etwas trocken und unmotiviert, wenn sie nicht zusammen mit einem Teil der konkreten Fragen präsentiert wird, aus denen sie entstanden ist. Das sind neben Fragen der Algebra selbst, wie der Lösungstheorie von algebraischen Gleichungen, vor allem Fragen der Geometrie und der Zahlentheorie.   

Die Algebra hat sich insbesondere aus dem Bedürfnis heraus entwickelt, für die Lösung von Gleichungen und Gleichungssystemen eine Theorie zu entwickeln. Nehmen wir die einfache Gleichung $2\cdot x = 1$. Wir erkennen sofort, dass diese für $x = \tfrac {1}{2}$ erfüllt ist. Ist dies die richtige Lösung?

Die Lösung für $x$ gilt nur, wenn wir in einem Zahlenbereich arbeiten, der „groß“ genug ist. Die rationalen Zahlen $\Q$ oder die reellen Zahlen $\R$ umfassen beispielsweise die Lösung $x=\tfrac{1}{2}$. Arbeiten wir allerdings im Zahlenbereich der natürlichen Zahlen $\N$ oder im Zahlenbereich der ganzen Zahlen $\Z$, dann hat obige Gleichung keine Lösung. Somit befassen wir uns in der Algebra auch mit Zahlenbereiche und deren Struktur. Die typischen Zahlenbereiche sind durch folgende Teilmengenkette gegeben:

\[\N \subseteq \Z \subseteq \Q \subseteq \R \subseteq \C\]

Algebra untersucht die Strukturen dieser Zahlenbereiche. Typisch an Zahlen ist, dass man mit ihnen rechnen kann. Eine Rechenoperation ist eine Verknüpfung, die zwei Zahlen nimmt und als Ergebnis eine neue Zahl zurückgibt. Eine solche Verknüpfung ist die Addition, die in allen Mengen der obigen Teilmengenkette definiert ist. Nehmen wir als Beispiel den Zahlenbereich $\Z$ der ganzen Zahlen. Dort gilt, dass die Gleichung $2 + x = 0$ für $x = -2$ eine Lösung hat. Jedoch gibt es in den natürlichen Zahlen $\N$ diese Lösung nicht, denn dort existieren keine negativen Zahlen.

Die lineare Algebra befasst sich mit der mathematischen Struktur des Vektorraums. Dieser ist durch Abstraktion der anschaulichen Vektorrechnung, wie sie in der Schule gelehrt wird, entstanden. Die so genannten linearen Abbildungen zwischen zwei gleichen Vektorräumen spielen bei der Untersuchung von Vektorräumen eine herausragende Rolle. Diese linearen Abbildungen können durch Matrizen auf eine einfache und anschauliche Weise beschrieben werden. Die Matrizenrechnung ist daher der Formalismus zum „Rechnen“ mit linearen Abbildungen. Die Entwicklung der linearen Algebra war unter anderem durch das Bedürfnis nach einer Lösungstheorie für lineare Gleichungssysteme motiviert. Dabei wird das Lösen linearer Gleichungssysteme auf das Lösen von Matrizengleichungen zurückgeführt. Auch in der analytischen Geometrie finden sich Methoden und Objekte aus der linearen Algebra wieder!

\section{Variablen in der Algebra}

Mit dem Einsatz von Variablen und Termen nähern wir uns der „reinen“ Mathematik. Wo es bisher um konkretes Rechnen mit Alltagsbezug ging (sei es beim Rechnen mit den Grundrechenarten, bei geometrischen Objekten, beim Dreisatz oder bei der Prozentrechnung) erfolgt nun ein erster Schritt in die Abstraktion. Einigen mag diese neue Mathematik als „abgehoben und leer“ und nicht mehr praxisrelevant vorkommen. Variablen und Terme ermöglichen aber eine eindeutige, genaue und sehr knappe symbolische Darstellung für (mathematische) Sachverhalte, bei denen auf der sprachlichen Ebene viele Sätze nötig wären.

\subsection{Grundvorstellungen zu einer Variablen}
Die unterschiedlichen Möglichkeiten, die der Umgang mit Variablen bietet, stellen die erste Schwierigkeit dar. Anders als im täglichen Leben sind Variablen keine Abkürzungen für irgendwelche Dinge, sondern letztlich ein Platzhalter für eine oder mehrere Zahlen, für alle Zahlen oder auch für einen Term.

Eine Variable ist also ein Name für eine Leerstelle in einem logischen oder mathematischen Ausdruck. Der Begriff leitet sich vom lateinischen Adjektiv variabilis (veränderlich) ab. Gleichwertig werden auch der schon erwähnte Begriff „Platzhalter“ oder „Veränderliche“ benutzt. Beim Zusammentreffen mehrerer Variabler mit funktionalen Zusammenhängen zwischen ihnen unterscheidet man abhängige und unabhängige Variablen. Alle unabhängigen Variablen gehören zum [[w:Definitionsmenge|Definitionsbereich]]. Die davon abhängigen Variablen gehören zum [[w:Zielmenge|Wertebereich]].

Variablen werden meist mit Buchstaben bezeichnet. Man kann Werte einsetzen (Einsetzungsaspekt) oder mit den Variablen rechnen (Kalkülaspekt). Variablen werden auch genutzt, um damit etwas zu beschreiben (Gegenstandsaspekt).

\subsection{Arten von Variablen}
Es gibt verschiedene Arten von Variablen<ref>vgl. auch [[w:Variable (Mathematik)|Variable]]</ref>:

\begin{enumerate}
\item ''Die Variable als unbekannter Wert einer gesuchten Zahl:'' Dieser Typ von Variablen ist wohl der bekannteste. Hier steht die Variable für eine zu ermittelnde Zahl. Der Wert für die Variable ist so zu bestimmen, dass eine vorgegebene Gleichung erfüllt ist und eine wahre Aussage entsteht. Beispiel: $3\cdot a + 4 = 13$, also ist die gesuchte Zahl $a = 3$.
\item ''Variable in Termen als Beweisprinzip:'' Diese Variablen kommen in einer allgemeinen Beschreibung eines Rechengesetzes oder einer geometrischen Formel vor. Für die Variablen können beliebige und für den Kontext sinnvolle Zahlen eingesetzt werden. Ein Beispiel sind die Variablen $a$ und $b$ in der binomischen Formel $(a+b)^2 = a^2 + 2\cdot a\cdot b + b^2$ oder die Variablen $A$ (Fläche), $g$ (Länge der Grundlinie) und $h$ (Höhe) in der Formel $A = \tfrac{1}{2} \cdot g\cdot h$ zur Flächenberechnung eines Dreiecks.
\item ''[[w:unabhängige Variable|Unabhängige Variable]]'': Wir sprechen von einer unabhängigen Variablen, falls ihr Wert innerhalb ihres Definitionsbereiches frei gewählt werden kann. Wenn man sich den Luftdruck $p$ in der Höhe $h$ anschaut, so ist $h$ eine unabhängige Variable.
\item ''[[w:abhängige Variable|Abhängige Variable]]'': Der Wert einer Variablen ist abhängig von den Werten anderer Variablen. So ist der Luftdruck $p$ eine abhängige Variable, abhängig von der Höhe, dargestellt mit dem Buchstaben  $h$.
\item ''[[w:Parameter (Mathematik)|Parameter]]'': Ein Parameter ist eine an sich unabhängige Variable, die aber zumindest in einer gegebenen Situation eher als eine festgehaltene Größe aufgefasst wird. Zum Beispiel ist der Bremsweg $s$ eines Fahrzeugs vor allem von dessen Geschwindigkeit $v$ abhängig, denn $s = f \cdot v^2$, dabei ist $f$ ein Parameter, dessen Wert bei genauerer Betrachtung von weiteren Parametern wie der Griffigkeit des Straßenbelags und der Profiltiefe der Reifen abhängig ist.
\item ''[[w:Mathematische Konstante|Konstanten]]'': Häufig werden auch konkrete ''unveränderliche'' Zahlen, ''festliegende'' Größen oder auch durch Messabweichungen ''unsichere'' bzw. ''unrichtige'' Messwerte mit einem Formelzeichen versehen, das nun statt der numerischen Angabe verwendet werden kann. Das Formelzeichen steht für den in der Regel unbekannten wahren Wert. Beispiele sind die Kreiszahl $\pi = 3{,}1415\ldots$ (je nachdem, wie viele Dezimalstellen ich zulasse, verändert sich der Wert der Kreiszahl $\pi$). Auch die Elementarladung $e = 1{,}602\ldots\cdot 10^{-19}\,\mathrm{As}$ ist eine durch Messungenauigkeiten nicht eindeutig bestimmte Zahl, also eine Variable, abhängig von der Messgenauigkeit.
\item ''[[w:Statistische Variable|Statistische Variable in der Stochastik]]'': Nehmen wir an, bei einer Wahl haben $p$ von $n$ Wähler eine bestimmte Partei gewählt. Die Statistische Variable $X$: ''Partei gewählt'' hat zwei mögliche Ausprägungen: ''Partei gewählt'' oder ''Partei nicht gewählt''. Die relative Häufigkeit, mit der die Partei gewählt wurde, ist $p/n$, und die, mit der die Partei nicht gewählt wurde, ist $1-p/n$.
\item ''[[w:Freie Variable und gebundene Variable|Freie und gebundene Variablen in der Logik]]'': Variablen sind gebunden, wenn sie in einer Notation als Hilfsvariable definiert werden: $\sum_{i=1}^{n}a_i$, dabei ist $i$ gebunden und $n$ und $a$ sind frei. In der Schreibweise $\int_a^b f(x)\,\mathrm dx$ ist $x$ gebunden und $a$, $b$ und $f$ sind freie Variablen.
\end{enumerate}

\section{Anwendungen der linearen Algebra}
In der Mathematik findet die lineare Algebra vielfältige Anwendungen. Die mehrdimensionale Analysis benötigt die lineare Algebra als Grundlage, da sie Abbildungen zwischen Vektorräumen analytisch untersucht. Auch ist es oft einfacher in linearen Systemen zu rechnen. Ein Beispiel hierfür sind lineare Differentialgleichungssysteme<ref>vgl. z.B. http://www.mathepedia.de/Homogene DGL Systeme.html</ref>.

Auch außerhalb der Mathematik finden sich Anwendungsbeispiele: Die Berechnung von Drehbewegungen ist ohne Vektorrechnung und ohne die Anwendung von Matrizen (z.B. für das [[w:Trägheitstensor|Trägheitsmoment]]) nicht durchführbar. Die [[w:Quantenmechanik|Quantenmechanik]] nutzt die lineare Algebra als ihr Fundament, um quantenmechanische Systeme zu beschreiben. 

In der Elektrotechnik benötigt man lineare Algebra zusammen mit mehrdimensionaler Analysis für die Beschreibung von elektrischen und magnetischen Feldern die [[w:Maxwell-Gleichungen|Maxwell-Gleichungen]]. Operatoren wie der [[w:Gradient|Gradient]] in krummlinigen Koordinaten sind ohne lineare Algebra nicht denkbar. 

Die komplexen Zahlen, die man in der Wechselstromlehre benötigt, können ebenfalls als 2-dimensionaler Vektorraum $\R^{2}$ beschrieben werden<ref>siehe dazu z.B. http://www.ate.uni-due.de/data/get12/GET2 4 Komplex Wechsel.pdf</ref>.

Auch die [[w:Betriebswirtschaftslehre|Betriebswirtschaftslehre (BWL)]] nutzt Methoden der linearen Algebra. Wenn zum Beispiel für verschiedene Endprodukte verschiedene Rohstoffe notwendig sind, kann man sich fragen, wieviele Endprodukte man mit einer gegebenen Rohstoffmenge produzieren kann. Diese Frage ist mit Methoden der linearen Algebra lösbar<ref>vgl. dazu https://www.mathematik.uni-muenchen.de/~renesse/LA09/labwl.pdf</ref>.

%%% Local Variables:
%%% mode: LaTeX
%%% TeX-master: "../mfnf"
%%% End:
