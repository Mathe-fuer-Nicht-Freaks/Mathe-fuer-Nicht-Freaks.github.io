\chapter{Gruppen}
\label{Gruppen}
Eine Gruppe ist eine algebraische Struktur mit einer inneren Verknüpfung. Intuitiv besteht eine Gruppe aus Operationen, die man rückgängig machen kann.
\section{Einleitung}
\subsection{Ein erstes Beispiel: Vertauschen von Steinen}
Stellen wir uns folgende Situation vor: Wir haben drei Steine, die in einer Reihe liegen. Diese Steine können wir umordnen, also ihre Reihenfolge verändern. Wir möchten jetzt dieses einfache Szenario etwas genauer untersuchen.  

Dazu müssen wir zunächst klären, was wir unter Umordnungen verstehen: 
Eine Umordnung ist eine Vorschrift, die angibt, von welcher Position aus ein Stein auf welche Position verlegt wird.
Nach einer Umordnung muss jeder Stein wieder auf einer Position liegen und keine Position darf doppelt besetzt sein. Dabei erlauben wir auch, dass Ausgangs- und Endposition eines oder mehrerer Steine gleich sind. Insbesondere betrachten wir "Nichts tun", also das Belassen aller Steine an ihrem Platz auch als eine Umordnung. Diese Umordnung spielt eine entscheidende Rolle. Deshalb geben wir ihr auch einen eigenen Namen; wir nennen sie die "Identitätstransformation".
Weitere Beispiele für Umordnungen sind  "Der linke und der rechte Stein tauschen die Plätze" oder "Der linke Stein wandert in die Mitte, der mittlere Stein wandert nach rechts, und der rechte nach links". 


Etwas formaler: mathematisch betrachtet bilden die Positionen der Steine die Menge $ \{ \text{links}, \text{mittig}, \text{rechts} \} $.
Wenn wir einen Stein bewegen, verschieben wir ihn von einem Element dieser Menge auf ein anderes. Wir können also Umordnungen der Steine als Abbildungen dieser Menge auf sich selbst auffassen.
Da nie zwei verschiedene Steine auf dieselbe Position verschoben werden dürfen (können), ist eine Umordnung (aufgefasst als Abbildung) injektiv; da jede Position wieder besetzt werden muss, ist sie außerdem surjektiv.
Bei unseren Umordnungen handelt es sich also um bijektive (injektive ''und'' surjektive) Abbildungen einer dreielementigen Menge auf sich selbst. Umgekehrt entspricht jede bijektive Abbildung $ f $ der Menge $ \{ \text{links}, \text{mittig}, \text{rechts} \} $ in sich selbst einer Umordnung. (Der Stein auf Position x wird unter der zu f korrespondierenden Umordnung auf die Position f(x) verschoben.)
Tatsächlich spielt lediglich die Größe der Menge (drei) mathematisch eine Rolle. Wir könnten genauso gut auch drei Stifte, Kaffeetassen, Radiergummis etc. vertauschen, oder andere Bezeichnungen für die zu besetzenden Positionen wählen (wie oben-Mitte-unten oder vorne-Mitte-hinten). Wir untersuchen also eigentlich Vertauschungen von drei unterscheidbaren Objekten. 


Was passiert, wenn wir Vertauschungen (Umordnungen) hintereinander ausführen?
Zum Beispiel können wir zuerst den linken mit dem mittleren Stein vertauschen, und anschließend den (neuen) mittleren Stein mit dem rechten.
In diesem Fall erhalten wir das gleiche Endergebnis, wie wenn wir in einem Schritt den linken Stein ganz nach rechts legen, und die anderen jeweils um eins nach links verschieben.
Diese Situation lässt sich in einer Grafik veranschaulichen (zur Verdeutlichung haben wir die Steine eingefärbt):
\begin{center}
  \includegraphics[width=15cm]{articles/imgs/Composing.pdf}
\end{center}
Wir behaupten nun, dass sich ''jede beliebige'' Hintereinanderausführung von zwei (oder sogar mehr) Umordnungen immer auch in einem einzigen Durchgang realisieren lässt.
In anderen Worten: Die Hintereinanderausführung von Umordnungen ergibt wieder eine Umordnung.

Dies überprüfen wir in zwei Schritten:

Zunächst müssen wir alle möglichen Umordnungen finden.
Bei drei Steinen ließe sich das noch mit etwas Herumprobieren bewerkstelligen. Wir wollen hier aber den systematischen Weg demonstrieren.
Dazu bauen wir die Umordnungen Stück für Stück auf: Als Erstes legen wir fest, auf welche Position der linke Stein bewegt wird. Wir haben drei Plätze zur Auswahl. Dann bewegen wir den mittleren Stein. Da keine Position doppelt belegt werden darf, stehen für jede mögliche Endposition des linken Steins noch zwei für den mittleren Stein zur Auswahl. Für den rechten Stein bleibt dann stets nur eine Möglichkeit übrig.
Insgesamt gibt es also $ 6 = 3 \cdot 2 \cdot 1 $ verschiedene Umordnungen dreier Steine.
Dies sind: Die Identitätstransformation, drei verschiedene Arten, jeweils zwei Steine zu vertauschen, und schließlich zwei Arten, einen der äußeren Steine „auf die andere Seite zu bringen“ (siehe Graphik unten).

Da wir nun alle Umordnungen kennen, müssen wir im zweiten Schritt alle möglichen Kombinationen durchprobieren.
Das Ergebnis können wir in einer Tabelle, ähnlich einer Einmaleins-Tabelle, darstellen. Dabei gibt die Kopfzeile an, welche Vertauschung zuerst, und die vordere Spalte an, welche Vertauschung daraufhin ausgeführt wird.
Das Ergebnis der Verknüpfung steht in der zugehörigen Zelle des Ergebnisbereichs:
[[File:S3 Cayley table.svg|frameless|center|upright=2.5||]]
Das eingefärbte Beispiel von oben finden wir beispielsweise in der dritten Spalte und zweiten Zeile des Ergebnisbereichs wieder.

{{:Mathe für Nicht-Freaks: Vorlage:Hinweis|Alternativ hätten wir auch so argumentieren können: Jede Umordnung entspricht einer bijektiven Abbildung $ \{ \text{links}, \text{mittig}, \text{rechts} \} \to \{ \text{links}, \text{mittig}, \text{rechts} \}$. Die Hintereinanderausführung von zwei Umordnungen entspricht der Verkettung der zugehörigen Abbildungen, und diese ist (als Verkettung bijektiver Abbildungen) bijektiv. Weil umgekehrt jede bijektive Abbildung  $ \{ \text{links}, \text{mittig}, \text{rechts} \} \to \{ \text{links}, \text{mittig}, \text{rechts} \}$ einer Umordnung entspricht, ist die Hintereinanderausführung von zwei Umordnungen wieder eine Umordnung. }}


Anhand der Tabelle kann man aber noch einige andere Dinge beobachten: 
Zum Beispiel sind die erste Zeile des Ergebnisbereichs und die Kopfzeile identisch.
Die Ergebnisse dieser Zeile entstehen, wenn man nach einer beliebigen Umordnung noch die Identitätstransformation durchführt. Da die Identitätstransformation die Position der Steine nicht ändert, ergibt sich insgesamt die zuerst durchgeführte Umordnung. Dies ist für uns nicht weiter verwunderlich, da wir mit dem zugrundeliegenden Szenario (Umordnungen von drei Steinen) vertraut sind und genau wissen, dass die Identitätstransformation „nichts tut“.

Um uns die wahre Bedeutung dieser Beobachtung vor Augen zu führen, stellen wir uns einmal vor, die Tabelle würde von jemandem betrachtet, dem wir nichts über das Szenario erzählt haben. So eine Person weiß nichts über die drei Steine und hat noch nie etwas von Umordnungen gehört.
Für diese Person sind die Einträge in der Tabelle einfach abstrakte Symbole, die ansonsten keine weitere Bedeutung tragen. Wir können uns in diese Situation versetzen, indem wir die Umordnungssymbole in der Tabelle durch beliebige andere Symbole ersetzen, die nichts mehr mit den drei Steinen zu tun haben.
Wir können zum Beispiel simple geometrische Formen wählen, einen Kreis für die Identitätstransformation und so weiter:
[[File:S3 Cayley table (abstract).svg|frameless|center|upright=1.7||]]

Natürlich sind auch in dieser Version der Tabelle die erste Zeile des Ergebnisbereichs und die Kopfzeile identisch. Es handelt sich ja um die ''gleiche'' Tabelle, nur mit anderen Symbolen. Allerdings sind Beobachtungen, die wir in der „neuen“ Tabelle machen, unabhängig von konkreten Eigenheiten des von uns gewählten Szenarios.

So gleichen sich nicht nur die ersten Zeilen, sondern auch die erste Spalte des Ergebnisbereichs und die vordere Spalte. Da wir aber nun nichts mehr über Steine und Identitätstransformationen wissen, kommen wir zu einem erstaunlichen Schluss: Wenn wir das Kreis-Symbol mit einem beliebigen Symbol kombinieren, egal ob an erster oder zweiter Stelle (sprich, in der ersten Spalte oder Zeile), ergibt sich in jedem Fall wieder das andere Symbol.
Ohne die Interpretation der Tabellenelemente als Umordnungen von Steinen ist diese Eigenschaft eine rein abstrakte Beobachtung, frei von jeder Begründung. Wir können lediglich anerkennen, dass es sich um eine grundlegende Eigenschaft des Kreis-Symbols zu handeln scheint.
Wir geben dieser Eigenschaft einen Namen: Neutralität. Wir sagen auch: „Das Kreis-Symbol ist ein '''neutrales Element''' (in dieser Tabelle)“.

Weiterhin können wir beobachten, dass in jeder Zeile und jeder Spalte des Ergebnisbereichs jedes Symbol ''exakt einmal'' vorkommt (ähnlich wie bei einem Sudoku).
Für die erste Zeile und Spalte ist das leicht nachzuvollziehen, aber es trifft auch auf alle anderen zu. Allerdings ist hier die Reihenfolge der Symbole (gegenüber der ersten Zeile/Spalte) verändert.
Dass tatsächlich ''alle'' Symbole in jeder Zeile und Spalte erscheinen, wird uns noch zu einem späteren Zeitpunkt in diesem Artikel beschäftigen.<!--ref group=link>Abschnitt zum Satz von Cayley etc. verlinken.</ref-->

Für den Moment wollen wir unsere Aufmerksamkeit aber auf ein spezielles Symbol richten, nämlich auf das neutrale Element (das Kreis-Symbol, die Identitätstransformation).
Wenn es im Ergebnisbereich auftaucht, können wir das nämlich folgendermaßen interpretieren:
Die zwei Symbole, die wir dazu verknüpft haben, verhalten sich in Kombination neutral.
Oder im Szenario der drei Steine: Die beiden Umordnungen ergeben hintereinander ausgeführt "Nichts tun". Die zweite Umordnung hebt die erste auf, kehrt sie um. 
Wir nennen ein Paar solcher Symbole, die zusammen das neutrale Element ergeben, "zueinander invers".
Eine Umkehrung eines Symbols bezeichnen wir als '''Inverses''' des Symbols.
Die Tatsache, dass das neutrale Element in jeder Zeile und Spalte des Ergebnisbereichs auftaucht, bedeutet, dass jedes Symbol ein Inverses hat.
Betrachten wir zum Beispiel das Fünfeck. Wir suchen das Fünfeck in der Kopfzeile und betrachten die darunter liegende Spalte des Ergebnisbereichs.
Dort finden wir das neutrale Element (den Kreis) an vierter Stelle.
Wenn wir die entsprechende Zeile nach vorne verfolgen, erkennen wir in der vorderen Spalte, dass das Karo invers zum Fünfeck ist.
Mit dem Sprachbeispiel "die Symbole sind invers zueinander" haben wir bereits angedeutet, dass diese Beziehung wechselseitig ist.
Und tatsächlich: Wenn wir in der Kopfzeile beim Karo starten und dieselbe Prozedur durchführen, enden wir in der vorderen Spalte beim Fünfeck.
Das Fünfeck ist also auch invers zum Karo, oder anders ausgedrückt: Karo und Fünfeck sind invers ''zueinander''.

Als weiteres Beispiel wollen wir das Dreieck betrachten. Wenn wir das Inverse des Dreiecks mit der gerade beschriebenen Methode ermitteln, erhalten wir wieder das Dreieck. Das Dreieck ist also sein eigenes Inverses. Solche Symbole nennen wir '''selbstinvers'''. Die Existenz solcher Symbole ist unproblematisch. Im Gegenteil, wenn wir wieder unser konkretes Szenario der drei Steine zurate ziehen, stellen wir fest, dass das Dreieck die Vertauschung des mittleren und rechten Steins symbolisiert. Verknüpfen wir das Dreieck mit sich selbst, tauschen wir die Steine also hin und zurück; das Ergebnis ist konsequenterweise die Identitätstransformation.

Uns fällt auf, dass die Tabelle nicht symmetrisch bezüglich der Diagonalen von oben links nach unten rechts ist.  Die Reihenfolge, in der wir zwei Symbole kombinieren, spielt also eine Rolle! Wenn wir zum Beispiel zuerst das Sechseck wählen und an zweiter Stelle das Karo, so ergibt sich das Quadrat. Setzen wir hingegen das Karo an erste Stelle und das Sechseck an die zweite, erhalten wir das Dreieck. Obwohl Inverse kommutieren, gilt also '''kein allgemeines Kommutativgesetz''' (für mehr Informationen siehe auch (todo)<!--ref group=link>Link zu abelschen Gruppen.</ref-->).

Bisher haben wir nur die Verknüpfung zweier Symbole und deren Ergebnis betrachtet. Da dieses Ergebnis aber wieder eines der Tabellensymbole ist, können wir es seinerseits mit einem dritten Symbol verknüpfen. Wir haben somit einen Weg gefunden, um ''drei'' Symbole miteinander zu kombinieren. 

Zum Beispiel können wir erst Viereck mit Karo verknüpfen (dabei erhalten wir das Fünfeck), und dann das Fünfeck mit dem Dreieck. Das ergibt das Fünfeck. Dabei müssen wir Acht auf die Reihenfolge geben. Wenn wir nur sagen: Wir verknüpfen Dreieck mit Karo mit Fünfeck, ist nicht klar, in welcher Reihenfolge wir die Verknüpfung ausführen. Wir könnten meinen: Zuerst verknüpfen wir Dreieck mit Karo, und das Ergebnis verknüpfen wir dann mit Fünfeck. Oder wir könnten meinen: Wir verknüpfen Dreieck mit dem Ergebnis der Verknüpfung von Karo und Fünfeck.

In beiden Fällen ergibt sich: 


[[File:S3 associativity evaluation.svg|frameless|center|upright=1.7||]]
{{todo|fix zooming behavior of vector graphics}}

Beide "Rechenwege" liefern also das gleiche Endergebnis. 

{{:Mathe für Nicht-Freaks: Vorlage:Hinweis|Die Bezeichnung der Verknüpfung mit "nach" ist motiviert durch das Szenario der drei Steine. Die rechts stehende Umordnung wird zuerst angewandt. ''Nach'' dieser folgt dann die links stehende Umordnung. Die Elemente werden also von rechts nach links miteinander verknüpft. Diese Reihenfolge ist wie bei der Verknüpfung von Abbildungen. }}

In dem von uns betrachteten Fall spielt es deshalb keine Rolle, welche Symbole wir zuerst zusammenfassen, bzw. wo wir die Klammern setzen:
[[File:S3 associativity equation.svg|frameless|center|upright=1.7||]]
{{todo|fix zooming behavior of vector graphics}}

Diese Beobachtung war nicht zufällig, sondern gilt für alle möglichen Kombinationen unserer Symbole. Ein vollständiger Beweis "zu Fuß" würde das Überprüfen von $6 \cdot 6 \cdot 6 = 216$ Dreierkombinationen erfordern, daher beschränken wir uns hier auf ein Beispiel. Du kannst weitere Kombinationen ausprobieren und wirst immer zu dem Ergebnis kommen, dass die Position der Klammern keine Rolle spielt, solange die Reihenfolge der drei Symbole selbst unverändert bleibt. Einen allgemeinen Beweis dieser Tatsache findest Du im <!--ref group=link>Link zum Abschnitt über symmetrische Gruppen.</ref--> 

Wir geben dieser Eigenschaft einen Namen: '''Assoziativität''', und sagen auch: "Die Hintereinanderausführung zweier Umordnungen / Verknüpfung zweier Symbole ist ''assoziativ''." Assoziativität bedeutet, dass wir beim Rechnen (in Gruppen) Klammern weglassen können.

\subsection{Was wir daraus lernen}
Nach all diesen Beobachtungen steht aber noch eine Frage im Raum: Wozu das Ganze? Warum haben wir gerade in großer Ausführlichkeit ein als ziemlich aus der Luft gegriffen erscheinendes Szenario untersucht?

Die Antwort darauf ist überraschend: Die Umsortierung der Steine hat in gewisser Hinsicht ähnliche Eigenschaften wie die Addition ganzer Zahlen, und es gibt sowohl im Alltag als auch in der Mathematik viele weitere Beispiele für Operationen, die diese Strukturmerkmale aufweisen. Das sich daraus ergebende Konzept wird in der Mathematik als '''Gruppe''' bezeichnet. Wir sagen: "Die Umordnungen dreier Steine und deren Hintereinanderausführung bilden eine Gruppe."
Gruppenstrukturen sind in der Mathematik und im Alltag allgegenwärtig. Einige Beispiele, in denen Gruppen eine entscheidende Rolle spielen, sind: Das Addieren von Vektorpfeilen in der Ebene, das Verdrehen eines Zauberwürfels, das Drehen eines Objekts in der Ebene/im Raum, die Multiplikation rationaler Zahlen, etc...

\section{Definition der Gruppe}

{{:Mathe für Nicht-Freaks: Vorlage:Definition
 |titel=Innere Verknüpfung
 |definition=

Sei $M$ eine nichtleere Menge. 

Eine innere Verknüpfung $ \boxdot $ auf $ M $ ist eine Abbildung $\boxdot: M \times M \to M, (m,n) \mapsto m \boxdot n $. 

}}

{{:Mathe für Nicht-Freaks: Vorlage:Hinweis| Man nennt $ M $ abgeschlossen unter der Verknüpfung $ \boxdot $, falls gilt: Für alle Elemente $m, n$ aus $M$ liegt $m  \boxdot  n $ wieder in $M$.
Eigentlich ist diese Bedingung redundant, denn sie sagt nichts weiter, als dass die Verknüpfung $ m \boxdot n $ zweier Elemente $ m,n $ aus $ M $ wieder ein Element $ k=m \boxdot n $ aus $ M $ ergibt, dies ist aber genau die definierende Eigenschaft der Verknüpfung, und daher eigentlich ohnehin erfüllt. Manchmal fordert man explizit die Abgeschlossenheit einer Menge $ M $ unter einer gegebenen Verknüpfung $ \boxdot $, um daran zu erinnern, die Wohldefiniertheit der Verknüpfung zu prüfen. Die Eigenschaft einer Menge, abgeschlossen unter einer Verknüpfung zu sein, ist insbesondere auch für das Konzept der Untergruppe relevant, welches wir später in diesem Artikel einführen werden. 
}}

{{: Mathe für Nicht-Freaks: Vorlage:Beispiel| 
|titel=Innere Verknüpfungen
|beispiel=
Beispiele für Verknüpfungen sind zum Beispiel die Multiplikation und Addition ganzer Zahlen (aufgefasst als Abbildungen von $ \mathbb{Z} \times \mathbb{Z} \to \mathbb{Z} $ oder die Hintereinanderausführung (Verkettung) von Abbildungen einer Menge $ N $ in sich selbst (aufgefasst als Abbildung $ F \times F \to F $, wobei $ F $ die Menge aller Abbildungen $ N \to N $ bezeichnet). 
}}

 
{{Anker|Definition}}

{{:Mathe für Nicht-Freaks: Vorlage:Definition
 |titel=Gruppe
 |definition=
Sei $ G $ eine nichtleere Menge und $ \boxdot: G \times G \to G $ eine innere Verknüpfung auf $ G $. 
Das Tupel $(G,\boxdot)$ wird '''Gruppe''' genannt, wenn es folgende Eigenschaften besitzt:

{{Liste
|type=ol
|item1='''Assoziativität:''' Die Verknüpfung $\boxdot$ ist assoziativ, das heißt: Für alle Elemente $a,b,c \in G$ gilt $a \boxdot (b \boxdot c)=(a \boxdot b) \boxdot c$  
|item2=''' Existenz eines neutralen Elements:''' Es existiert ein Element $e \in G$, sodass für alle Elemente $a \in G$ gilt: $e \boxdot a = a = a \boxdot e$. Dieses Element nennt man das neutrale Element von $G$ bezüglich der Verknüpfung $ \boxdot $. 
|item3='''Existenz inverser Elemente:''' Für jedes Element $a \in G$ gilt: Es existiert ein Element $a^{-1} \in G$, sodass $a^{-1} \boxdot a = e = a \boxdot a^{-1}$. Das Element $ a^{-1}$ mit dieser Eigenschaft heißt inverses Element oder Inverses von $ a$ unter der Verknüpfung $ \boxdot $. 
}}
}}
{{:Mathe für Nicht-Freaks: Vorlage:Hinweis|
Das neutrale Element ist, wie wir später sehen werden, eindeutig bestimmt. Eine Gruppe kann also nicht mehrere neutrale Elemente besitzen.

Außerdem gilt: Zu jedem Element $ a \in G $ existiert genau ein (unter der Verknüpfung $ \boxdot $ ) inverses Element $ a^{-1} \in G$. Man spricht auch von der Eindeutigkeit inverser Elemente. 
Die Eindeutigkeit ist nicht Teil der Definition, sondern folgt aus der definierenden Eigenschaft des neutralen bzw. der inversen Elemente. Manchmal wird die Eindeutigkeit in der Definition gefordert, dies ist aber nicht notwendig. 
}}
{{:Mathe für Nicht-Freaks: Vorlage:Hinweis|

Oft sagt man nur "$G$ ist eine Gruppe" anstatt "$(G, \boxdot)$ist eine Gruppe", falls man davon ausgeht, dass "klar" ist, welche Verknüpfung gemeint ist.
}}

\section{Definition Abelsche Gruppe}
{{:Mathe für Nicht-Freaks: Vorlage:Hinweis|
Wir haben in der Definition der Gruppe nicht gefordert, dass $\boxdot$ eine kommutative Verknüpfung ist. Es ist also erlaubt, dass Elemente $a, b \in G$ existieren mit $a \boxdot b \neq b \boxdot a$. Ein Beispiel dafür sind die Umordnungen von drei Steinen aus der Einleitung. Gruppen, bei denen die Verknüpfung kommutativ ist, haben besonders schöne Eigenschaften, weswegen sie ihren eigenen Namen bekommen:
}}

{{Anker|Definition}} 
{{:Mathe für Nicht-Freaks: Vorlage:Definition
 |titel=Abelsche Gruppe
 |definition=
Sei $(G, \boxdot)$ eine Gruppe.

Wir nennen $G$ '''abelsch''' (oder '''kommutativ'''), falls $\boxdot$ kommutativ ist, d.h. falls für alle $a, b \in G$ gilt: $a \boxdot b = b \boxdot a$
}}

{{:Mathe für Nicht-Freaks: Vorlage:Hinweis|
Abelsche Gruppen sind nach dem norwegischen Mathematiker Niels Henrik Abel (1802-1829) benannt.
}}

\section{Beispiele}

\subsection{Addition ganzer Zahlen}
{{:Mathe für Nicht-Freaks: Vorlage:Beispiel
 |titel=Die ganzen Zahlen unter Addition

 |beispiel=
Die ganzen Zahlen bilden mit der (gewöhnlichen) Addition eine Abelsche Gruppe. 

Für alle ganzen Zahlen $ a,b \in \Z $ gilt: $ a+b \in \Z $. Die ganzen Zahlen sind also abgeschlossen unter der Addition. 

\# Die Addition ist assoziativ und kommutativ. 
\#Die Null ist eine ganze Zahl, und das neutrale Element der Addition. 
\#Zu jeder ganzen Zahl $ a \in \Z $ ist $ -a $ wieder eine ganze Zahl, und das additive Inverse, da $ a+(-a)=0$
}}

\subsection{Kein Beispiel: Subtraktion ganzer Zahlen}
{{:Mathe für Nicht-Freaks: Vorlage:Beispiel
 |titel=Ganze Zahlen unter Subtraktion

 |beispiel=
Die Subtraktion $ -: \Z \times \Z \to \Z, (x,y) \mapsto x-y $ 
definiert eine innere Verknüpfung auf den ganzen Zahlen, da für $ x,y \in \Z $ auch immer $ x-y \in \Z $ gilt. 
Die ganzen Zahlen bilden aber keine Gruppe unter Subtraktion. 
Dies liegt daran, dass die Subtraktion nicht assoziativ ist. Denn für ganze Zahlen $ x,y,z \in \Z $ gilt $ x-(y-z)=x-y+z $, aber $ (x-y)-z =x-y-z $. Insbesondere ist für $ z \in \Z, z \neq 0 $ stets $ (x-y)-z \neq x-(y-z) $. 

Es lässt sich auch kein neutrales Element bezüglich der Subtraktion finden: Zwar gilt für 
$ x \in \Z $, dass $ x-0 =x $, aber $ 0-x =-x $, das heißt die Null ist kein neutrales Element. Wegen $ x-y \neq x $ für alle $ x,y \in \Z, y \neq 0 $ kann es kein neutrales Element geben. 
}}

{{:Mathe für Nicht-Freaks: Vorlage:Hinweis|
 Man nennt die Null im obigen Kontext rechtsneutrales Element, da für alle $ x \in \Z $ gilt $ x-0=x $. Die Null wirkt also von rechts wie ein neutrales Element. 
  }}

\subsection{Addition und Multiplikation rationaler Zahlen}
{{:Mathe für Nicht-Freaks: Vorlage:Beispiel
 |titel=Die rationalen Zahlen $ \Q $ unter Addition und Multiplikation
 |beispiel=
Die rationalen Zahlen $ \Q= \left\{\frac{a}{b} \ | \ a \in \Z, b \in \Z \setminus \{0 \} \right\}$ bilden mit der gewöhnlichen Addition eine Abelsche Gruppe, denn es gilt: 

$\frac{a}{b}+\frac{c}{d}=\frac{ad+bc}{bd} \in \Q $ für alle $ \frac{a}{b}, \frac{c}{d} \in \Q $. Genauer: Für ganze Zahlen $ a,b,c,d $ mit $b,d \neq 0 $ gilt immer $ ad+bc \in \Z $, und $ bd \in \Z \setminus \{0\} $, also auch $\frac{a}{b}+\frac{c}{d}=\frac{ad+bc}{bd} \in \Q $. Die rationalen Zahlen sind also abgeschlossen unter Addition. Anders ausgedrückt: Die gewöhnliche Addition (rationaler Zahlen) kann man als Abbildung $ +: \Q \times \Q \to \Q, (x,y) \mapsto x+y $ auffassen. Die Addition rationaler Zahlen ist also eine innere Verknüpfung auf $ \Q $. 
\#Die Addition ist assoziativ. 
\#Die Null ist das neutrale Element bezüglich der Addition, denn für alle $ q \in \Q $ gilt: $ q+0=q $. 
\#Zu jeder rationalen Zahl $ q \in \Q $ ist $ -q \in \Q $ das (unter Addition) inverse Element, da $ q+ (-q)=0 $. 

Weil die Addition rationaler Zahlen kommutativ ist, ist $ (\Q,+) $ folglich eine Abelsche Gruppe. 


Unter der Multiplikation bilden die rationalen Zahlen ohne Null ($ \Q\setminus \{0 \} $) eine Abelsche Gruppe. Tatsächlich:
\#Für zwei rationale Zahlen  $0 \neq \frac{a}{b}, \frac{c}{d} \in \Q $ (also $ a,b,c,d \in \Z \setminus \{0\} $) gilt immer $ \frac {a}{b} \cdot \frac{c}{d}= \frac {ac}{bd} \in \Q \setminus \{ 0 \}$, weil $ ac, bd \in \Z \setminus \{0\} $. Das heißt die Multiplikation $ \cdot: \Q \times \Q \to \Q, \left(\frac{a}{b}, \frac{c}{d}\right) \mapsto \frac{ac}{bd} $ ist eine innere Verknüpfung auf $ \Q $. 
\#Die Multiplikation ist assoziativ und kommutativ. 
\#Die $1$ ist das neutrale Element bezüglich der Multiplikation, denn für alle rationalen Zahlen $ x=\frac{a}{b}, a,b \in \Z , b \neq 0 $ gilt: $ 1\cdot x= \frac{1}{1} \cdot \frac{a}{b}= \frac {1\cdot a}{1 \cdot b}=\frac{a}{b} =x$, und genauso $ x \cdot 1=x $.   
\#Für $ \frac {a}{b} \in \Q \setminus \{0 \} $ ist $ \frac {b}{a} \in \Q $, weil $ a,b \in \Z $ und $ a \neq 0 $ und es gilt $ \frac{a}{b} \cdot \frac{b}{a}=\frac {b}{a} \cdot \frac {a}{b} =\frac{ba}{ab}=1 $. Das heißt $ \frac{b}{a} $ ist das zu $ \frac{a}{b} $ unter Multiplikation inverse Element. Die Null muss ausgeschlossen werden, weil es zur Null kein multiplikatives Inverses in $ \Q $ gibt. Für alle rationalen Zahlen $ q\in \Q $ gilt: $ q \cdot 0 =0 \neq 1 $. Die Multiplikation $ \cdot : \Q \times \Q \to \Q, (x,y) \mapsto x \cdot y $, wäre zwar eine wohldefinierte Abbildung, also eine Verknüpfung auf den rationalen Zahlen. Diese Multiplikation ist assoziativ und $ 1 $ ist ein neutrales Element, aber $ \Q $ ist mit der Multiplikation KEINE Gruppe, da Null kein inverses Element unter Multiplikation in $ \Q $ hat.  
}}

\subsection{Modulorechnung}
{{:Mathe für Nicht-Freaks: Vorlage:Beispiel
 |titel=Modulorechnung

 |beispiel=
Wer kennt das nicht. Du bist einkaufen, fünf kleine Milchpackungen für je 0,67€, Äpfel für 2,33€, ein Comicheft für 4,79€ und einen Toaster für 42,55€. Nun, wer kann Kleingeld leiden? Somit ist es ein natürliches Ziel, den passenden Centbetrag bereits im Voraus parat zu haben. Und so beginnen wir zu rechnen: $5 \cdot 0,67 + 2,33 + 4,79 + 42,55 = ... 53,02$. Uff, das wäre viel Wechselgeld geworden. Nun stellt sich aber die Frage, geht es nicht einfacher? Wenn uns die Eurobeträge gar nicht interessieren, wieso sollen wir sie dann mitschleppen? Einfacher ist es, nur die Centbeträge zu addieren. Eine Rechnung im Kopf würde dann eher so aussehen:  $5 \cdot 0,67= 3,35$, d.h. wir haben $35$ Cent. $0,35+0,33+0,79=0,68+0,79=1,47$, also $47$ Cent. Jetzt kommt noch der Toaster: $0,47+0,55=1,02$, also ist der endgültige Centbetrag 2. Schon viel handlicher.

Das mathematische Modell dahinter sind die sogenannten Restklassengruppen: Wir wollen wissen, welchen ganzzahligen Rest wir bei der Division einer Summe von ganzen Zahlen durch ein festes $n \in \N$ erhalten. Ein Beispiel für Restklassengruppen kennen wir schon aus der Einleitung: Das Rechnen auf der Uhr. Im obigen Beispiel interessiert uns der Rest bei Division des Preises (in Cent) durch $100$. Allgemeiner kann man sich auch für den Rest bei der Division durch eine beliebige ganze Zahl $ n $ interessieren. Im Folgenden berechnen wir mit $[a]$ für $a \in \Z $ die Äquivalenzklasse von $ a $ modulo $ n \Z $, also die Menge aller ganzen Zahlen, die bei Division mit Rest durch $n$ den gleichen Rest wie $a$ (bei Division durch $ n $ ) ergeben. Für eine ganze Zahl $ a \in \Z $ gilt stets $ a= q \cdot n+r $ für zwei eindeutig bestimmte ganze Zahlen $ q,r $ mit $ 0 \leq r \leq n-1 $. 
Sei $ a $ nun eine beliebige, feste ganze Zahl und $ r $ der Rest von $ a $ bei Division durch $n $. Damit können wir schreiben: $ [a]=  \{q\cdot n+r | q \in \Z \} $. Hilfreich ist auch die folgende Beschreibung: Für $ a \in \Z $ gilt: $ [a]= \{x \in \Z | n \text{ teilt } a-x \} $. 

Denn für $ x \in [a] $ gilt: Es gibt ein $ q \in \Z $, sodass $x= q \cdot n+r $. Weil $ r $ der Rest der Division von $ a $ durch $n $ ist, finden wir $ s \in \Z $, sodass $ a= s \cdot n+r $. Daraus folgt $ a-x= s \cdot n+r-(q \cdot n+r)= (s-q) \cdot n $ ist durch $ n $ teilbar. Umgekehrt gilt für eine ganze Zahl $ x \in \Z $ mit der Eigenschaft, dass $ a-x $ durch $ n $ teilbar ist: Es gibt ein $ q \in \Z $, sodass $ a-x= q\cdot n $. Daraus folgt $ x= a-q \cdot n=(s-q)\cdot n +r $, also $ x \in [a] $. 

Für ein Element $ x \in [a] $ gilt $ [x]=[a] $. 
Tatsächlich: Seien $ a,x $ ganze Zahlen mit $ x \in [a] $. 
Dann gilt für $ b \in [a]: b-x=(b-a)-(x-a) $ ist als Differenz zweier durch $ n $ teilbarer Zahlen durch $ n $ teilbar, also $ b \in [x]$. Umgekehrt gilt für $ b \in [x]: b-a=(b-x)+(x-a) $ ist als Summe zweier durch $n$ teilbarer Zahlen durch $ n $ teilbar, also $ b \in [a]$. 

Auf den ganzen Zahlen ist durch die Addition eine Gruppenstruktur gegeben. Wir zeigen jetzt, dass diese mit dem Rechnen modulo $n$ "verträglich" ist. Damit ist gemeint: Für zwei ganze Zahlen $ z_1,z_2 \in \Z $ mit $ z_1 \in [a], z_2 \in [b] $ gilt: $ z_1+z_2 \in [a+b] $. Damit können wir eine Addition von Äquivalenzklassen definieren via $ [a]+[b]= [a+b] $.  

Weil $ z_1 \in [a] $ finden wir ein $ k_1 \in \Z $, sodass $ z_1=k_1\cdot n +a $. Ebenso finden wir $k_2 \in \Z $, sodass $ z_2=k_2\cdot n + b $. Es folgt $ z_1+z_2 = n \cdot (k_1+k_2)+ a+b $. 
Sei $ c \in \{0,...,n-1 \} $, sodass $z_1+z_2 \in [c]$. Dann gilt $ z_1+z_2=n \cdot k_3 +c $ für ein $ k_3  \in \Z$. Also ist $ z_1+z_2= n \cdot k_3+c=n \cdot(k_1+k_2)+a+b \implies a+b= n \cdot (k_3-(k_1+k_2)) + c $. Das heißt, $ a+b $ ergibt bei Division durch $ n $ den Rest $ c $. Deswegen ist $ a+b \in [c] $, also $ [a+b]=[c] $. Es gilt somit $z_1+z_2 \in [a+b] $. 
Deswegen ist die Addition modulo $ n \Z $ wohldefiniert, beziehungsweise die Addition ganzer Zahlen mit dem Rechnen modulo $n $ "verträglich". Die Assoziativität der Addition ganzer Zahlen überträgt sich auf das Addieren modulo $ n \Z $. 

Die Äquivalenzklasse $[0] $ (die aus allen durch $ n $ teilbaren Zahlen besteht) ist das neutrale Element bezüglich dieser Addition. 

Für $ a \in \Z $ ist $[-a]=[n-a]$ das inverse Element. Damit bildet $ \Z / n \Z $ bezüglich der oben definierten Addition eine Gruppe.  

}}

\subsection{Die triviale Gruppe}
{{:Mathe für Nicht-Freaks: Vorlage:Beispiel
 |titel=Die triviale Gruppe
 |beispiel= Auf einer einelementigen Menge $ M = \{m \} $ gibt es nur eine Möglichkeit, eine innere Verknüpfung $ \boxdot $ zu definieren, und diese lautet $ m \boxdot m=m $. Jede einelementige Menge wird unter der Verknüpfung $ \boxdot: M \times M \to M , (m,m) \mapsto m $ zu einer Gruppe. 

 Denn es gilt:
 \begin{enumerate}
 \item $ m \boxdot \underbrace{(m \boxdot m)}_{=m}=m \boxdot m=m $ und $ \underbrace{(m \boxdot m)}_{=m}\boxdot m =m $, das heißt, die Verknüpfung ist assoziativ. 

\item $ m $ ist das neutrale Element, weil $ m \boxdot m= m $. 
\item $ m $ ist zu sich  selbst invers. 
 \end{enumerate}

}}

\section{Eigenschaften von Gruppen}
\subsection{Lösbarkeit von Gleichungen}
Man kann sich fragen, unter welchen Bedingungen eine Gleichung der Form $a \boxdot x = b$ mit Elementen $a, b $ einer Gruppe $ G $ eine (eindeutige) Lösung $ x \in G $ hat. Tatsächlich gilt: Für zwei Elemente $ a, b $ einer Gruppe $(G, \boxdot)$ existiert immer genau ein Element $ x \in G $, sodass $ a \boxdot x=b $.

Dass dies so ist, ergibt sich aus den folgenden Umformungen:  
\begin{align*} 
a \boxdot x &= b\\[0.3em]
&{\color{green}\left\Downarrow \text{auf beide Seiten } (a^{-1} \boxdot \dots) \text{ anwenden} \right.}\\[0.3em]
a^{-1} \boxdot (a \boxdot x) &= a^{-1} \boxdot b \\[0.3em]
&{\color{green}\left\downarrow \text{Assoziativität der Operation } \boxdot \right.}\\[0.3em]
(a^{-1} \boxdot a) \boxdot x &= a^{-1} \boxdot b \\[0.3em]
&{\color{green}\left\downarrow \text{Definition des Inversen zu } a \right.} \\[0.3em] 
e_G \boxdot x &= a^{-1} \boxdot b \\[0.3em]
&{\color{green}\left\downarrow e_G \text{ ist neutrales Element} \right.} \\[0.3em]
x &= a^{-1} \boxdot b. \\[0.3em]
\end{align*}

Falls also $ x \in G $ eine Lösung der Gleichung $ a \boxdot x=b $ ist, so muss gelten $ x= a^{-1} \boxdot b $. Das heißt, die Gleichung kann höchstens eine Lösung, nämlich $ x= a^{-1} \boxdot b $ in $ G $ haben. Einsetzen ergibt, dass $ x= a^{-1} \boxdot b $ die Gleichung löst, weil 
\begin{align*}  & a \boxdot (a^{-1} \boxdot b)\\[0.3em]
&{\color{green}\left\downarrow \text{Assoziativität der Operation } \boxdot \right.}\\[0.3em]
=&(a\boxdot a^{-1}) \boxdot b  \\[0.3em]
&{\color{green}\left\downarrow \text{Definition des Inversen/neutralen Elements } \right.} \\[0.3em] 
=& e_G \boxdot b =b. \\[0.3em]
\end{align*}

Alternativ hätte man auch sehen können, dass man die Implikation $ a \boxdot x=b \Rightarrow a^{-1} \boxdot(a \boxdot x)= a^{-1} \boxdot b $ durch Multiplikation von links mit $ a $ auf beiden Seiten umkehren kann. Es gilt deswegen tatsächlich $ a \boxdot x=b \Leftrightarrow a^{-1} \boxdot(a \boxdot x)= a^{-1} \boxdot b $, und die rechte Seite ist wegen Assoziativität von $\boxdot $ bzw. der Definition von inversen/ neutralen Elementen äquivalent zu $ x= a^{-1} $.  Das heißt, $ x= a^{-1} \boxdot b $ ist eine, und die einzige Lösung der Gleichung $ a \boxdot x=b $. 

Wir haben nur die Eigenschaften einer Gruppe (und keine Zusatzannahmen über die Art der Gruppenstruktur) verwendet, um einen Lösungsausdruck für $x$ als Funktion von $ a$ und $ b $ herzuleiten. Deshalb ist in allen Gruppen die Gleichung $ a \boxdot x= b $ eindeutig lösbar.
Die eindeutige Lösbarkeit ist tatsächlich etwas Besonderes, denn es gibt einige uns bekannte Verknüpfungen, die diese Eigenschaft nicht erfüllen. Zum Beispiel gibt es keine ganze Zahl $ x \in \Z $, sodass $ 2 n\cdot x=1 $. Das heißt, die Gleichung $ 2 \cdot x=1 $ hat keine Lösung in den ganzen Zahlen $ \Z $. Insbesondere kann $ (\Z, \cdot) $ deshalb keine Gruppe sein. Dass die Gleichung keine Lösung hat, liegt daran, dass die $ 2 $ kein multiplikatives Inverses in den ganzen Zahlen hat. 
Die Gleichung $ 0 \cdot x=0 $ ist wahr für alle rationalen Zahlen $ x $, hat also in $ \Q $ unendlich viele Lösungen. Deshalb kann $ (\Q, \cdot) $ keine Gruppe sein. Bemerke: $ \Q \setminus \{0\} $ wird mit der gewöhnlichen Multiplikation zu einer Gruppe, und für rationale Zahlen $ a,b \in \Q \setminus \{0 \} $ hat die Gleichung $ a \cdot x= b $ immer genau eine Lösung in $ \Q $, nämlich $ x= \frac{b}{a} $. 

Wichtig ist, dass falls $(G, \boxdot)$ eine Gruppe ist, die oben gemachten Umformungen ''immer'' möglich sind. Unabhängig von der konkreten Wahl für $a$ und $b$ '''existiert''' also immer '''genau eine''' Lösung $ x \in G $ der Gleichung $ a \boxdot x= b $, nämlich $ x= a^{-1} \boxdot b $. Man kann also die Gleichung nach $ x $ umstellen/auflösen. 

Eine hilfreiche Anwendung davon ist der folgende Spezialfall: In Gruppen ist "kürzen" erlaubt: Für Elemente $ a,x,y $ einer Gruppe $ (G, \boxdot) $ gilt: $a \boxdot x = a \boxdot y \implies x = y$. 

Ganz ähnlich (jedoch mit Anwendung von $(... \boxdot a^{-1})$ von rechts) können wir auch Kürzbarkeit auf der rechten Seite zeigen: $x \boxdot a = y \boxdot a \implies x = y$. Aber Vorsicht: In der Gleichung $a \boxdot x = y \boxdot a$ kann man im Allgemeinen nicht $ a $ kürzen, da $a$ hier auf verschiedenen Seiten auftritt. Die Gruppen, in denen dies funktioniert, sind genau die Abelsche Gruppen, da dort $ g \boxdot h = h \boxdot g $ gilt.

\subsection{Eindeutigkeit des neutralen Elements}
{{:Mathe für Nicht-Freaks: Vorlage:Satz
 |titel=Eindeutigkeit des neutralen Elements
 |satz=
Gruppen besitzen genau ein neutrales Element, das bedeutet: Für jede Gruppe $(G, \boxdot)$ gibt es genau ein $e \in G$ sodass für alle $a \in G$ gilt $a \boxdot e = a = e \boxdot a$

 |beweis=
Sei $(G, \boxdot)$ eine Gruppe. Nach Definition einer Gruppe gibt es ein neutrales Element $ e \in G $. Angenommen, $ e' \in G $ ist ein weiteres neutrales Element von $(G, \boxdot)$. Per Definition eines neutralen Elementes haben wir für alle $ a \in G $ : 

\begin{align*} a &= e \boxdot a = a \boxdot e \\ a &= e' \boxdot a = a \boxdot e' \end{align*}

Insbesondere können wir $ a=e' $ in die erste Formel und $ a = e $ in die zweite Formel einsetzen. Das ergibt:

\begin{align*} e' = e \boxdot e' = e' \boxdot e \\ e = e' \boxdot e = e \boxdot e' \end{align*}

Das liefert schon $e' = e \boxdot e' = e$. Also gibt es nur ein neutrales Element, nämlich $ e $.
}}

\subsection{Eindeutigkeit der inversen Elemente}
Auch inverse Elemente sind eindeutig:

\label{Gruppen:EindeutigkeitInverse}
{{:Mathe für Nicht-Freaks: Vorlage:Satz
 |titel=Eindeutigkeit von inversen Elementen
 |satz=Sei $(G, \boxdot)$ eine Gruppe, $g \in G$ ein Element. Dann gibt es genau ein $h \in G$ mit $g \boxdot h = e_G = h \boxdot g$
 |beweis=Die Existenz eines Elements $ G $ sodass $g \boxdot h = e_G = h \boxdot g$ haben wir in der Definition der Gruppe gefordert.
 
Zur Eindeutigkeit von g: Angenommen, es gibt $h_1, h_2 \in G$, mit $h_1 \boxdot g = g \boxdot h_1 = e_G $ und $ h_2 \boxdot g = g \boxdot h_2 = e_G$.
Dann folgt 

\begin{align*} 
 h_1 = h_1 \boxdot e_G=h_1 \boxdot (g \boxdot h_2 )=&\\[0.3em]
&{\color{green}\left\downarrow\text{Assoziativität von } \boxdot \right.}\\[0.3em]
=&(h_1 \boxdot g) \boxdot h_2= e_G \boxdot h_2 = h_2 \\[0.3em]
\end {align*}

Das inverse Element $ g^{-1} $ eines Elements $ g \in G $ (unter $ \boxdot $) ist also eindeutig bestimmt.


}}

\subsection{Potenzgesetze}
Außerdem gelten die üblichen Potenzgesetze: Wenn wir $ a^k:=\underbrace{a \boxdot a \boxdot a ... \boxdot a}_{k mal} $ und $ a^{-k}:= \underbrace{a^{-1} \boxdot a^{-1} ... \boxdot a^{-1}}_{k mal} $ für $k \in \mathbb N$ definieren. Es gilt also für alle $k, l \in \Z$ und $ a \in G $: $ a^k \boxdot a^l=a^{k+l} $ und $ (a^k)^l=a^{kl} $. 
Insbesondere ist das inverse Element von $ a^{-1} $ $ (a^{-1})^{-1}=a^{(-1)\cdot (-1)}=a$, für alle Elemente $ a \in G $. Jedes Element ist  also invers zu seinem eigenen inversen Element. Diese Eigenschaft gilt allgemein in allen Gruppen, Kommutativität der Verknüpfung ist dafür nicht erforderlich.  
{{todo| Potenzgesetze beweisen}}

{{:Mathe für Nicht-Freaks: Vorlage:Hinweis|

Häufig verwendete andere Schreibweisen für das neutrale Element sind $1, 1_G$, oder $0, 0_G $. Letztere werden vorrangig für Abelsche Gruppen verwendet. Eine sehr bekannte Abelsche Gruppe bilden die ganzen Zahlen unter Addition, die Null ist dort das neutrale Element. In Anlehnung daran schreibt man die Verknüpfung einer Abelschen Gruppe häufig "additiv", man verwendet also eines der Symbole $ \boxplus, +$ an Stelle von $\boxdot$, und $0, 0_G $ für das neutrale Element, sowie $ -a $ statt $ a^{-1} $ für das Inverse zu einem Element $ a \in G $.
}}

\section{Aufgaben}
{{:Mathe für Nicht-Freaks: Vorlage:Aufgabe
|titel= Die natürlichen Zahlen sind keine Gruppe (weder unter Addition noch unter Multiplikation)
|aufgabe=Zeige, dass die natürlichen Zahlen weder unter der gewöhnlichen Addition, noch unter Multiplikation eine Gruppe bilden.
 |lösung=
\#Zur Addition: In $ \N_1 $ gibt es kein neutrales Element, da für alle $ m,n \in \N_1 $ gilt: $ m+n > m $. Daher kann man auch nicht die Existenz von Inversen untersuchen. (Weil nicht definiert ist, wann zwei Elemente zueinander invers sind). In $ \N_0 $ ist die $ 0 $ ein neutrales Element unter Addition, weil für alle $ m \in \N $ gilt: $ m+0=m$. Aber (mit Ausnahme der Null) gibt es keine natürliche Zahl, die ein inverses Element in den natürlichen Zahlen hat. Denn für alle $ m \in \N_1, n \in \N_0 $gilt: $ m+n \geq m > 0 $. 
\#Zur Multiplikation: Die natürlichen Zahlen sind abgeschlossen unter Multiplikation, die Multiplikation ist assoziativ und mit der $ 1 $ haben wir ein neutrales Element der Multiplikation. Aber: Die $ 1 $ ist die einzige natürliche Zahl, die ein multiplikativ Inverses in den natürlichen Zahlen besitzt. 
Das kann man zum Beispiel wie folgt zeigen: 
Für alle $ m,n \in \N_1 , m \neq 1$ gilt: $ m \cdot n \geq m > 1 $. Also hat keine natürliche Zahl $ m \in \N_1, m >1 $ ein multiplikatives inverses Element in $ \N_1 $. Weil $ m \cdot 0 =0 \neq 1 $ hat keine natürliche Zahl $ m > 1 $ ein multiplikatives Inverses in $ N_ 0 $. Ebenso hat die  Null kein multiplikatives inverses Element (in $ \N_0 $). 

Alternativ kann man auch so argumentieren: Es gilt $  \N_1 \subset \Q \setminus \{0\} $ und die Multiplikation natürlicher Zahlen ist genau die Einschränkung der Multiplikation rationaler Zahlen auf die natürlichen Zahlen. Wie wir oben gezeigt haben, sind die inversen Elemente in einer Gruppe eindeutig bestimmt, und für eine rationale Zahl $ q \in \Q \setminus \{0\} $ ist $ \frac{1}{q} $ ihr multiplikatives Inverses. Weil für $ m \in \N, m >1 $ gilt $\frac{1}{m} \notin \N $ können natürliche Zahlen $n $ aus $ \N_1 $ kein multiplikatives Inverses haben. (Wenn zwei Elemente $m,n \in \N_1 $ zueinander invers wären, wären sie auch aufgefasst als Elemente von $ (\Q \setminus \{0\}, \cdot) $ zueinander invers. )
Für $ \N_0 $ kann man die Aussage zeigen, indem man (wie oben) die Null separat betrachtet. 
}}
{{:Mathe für Nicht-Freaks: Vorlage:Aufgabe
 |titel=Die ganzen Zahlen sind keine Gruppe unter Multiplikation
 |aufgabe=Überlege Dir, wieso $ \Z \setminus \lbrace 0 \rbrace$ mit der gewöhnlichen Multiplikation keine Gruppe bildet.
 |lösung= Die Multiplikation ganzer Zahlen ist eine innere Verknüpfung, weil für $ a,b \in \Z $ gilt $ a \cdot b \in \Z $. Die $ 1 $ ist ein neutrales Element. In einer Gruppe müssen zu jedem Element Inverse existieren. 
Aber es gibt keine ganze Zahl $ z \in \Z $ mit $ z \cdot 2 = 1$. Daher kann $ \Z \setminus \lbrace 0 \rbrace $ keine Gruppe sein. Genauer: $ 1, -1 $ sind die einzigen ganzen Zahlen, die multiplikative Inverse in den ganzen Zahlen haben. Dies kann man zum Beispiel so zeigen: Falls $ z_1,z_2 \in \Z $ zueinander invers sind, gilt $ 1= z_1 \cdot z_2= |z_1 \cdot z_2 |= |z_1|\cdot |z_2 | $. Weil $ |z_1|, |z_2| \in \N_{0} $ folgt daraus (siehe Aufgabe "Die natürlichen Zahlen bilden unter Multiplikation keine Gruppe, oben), dass $ |z_1|=|z_2|=1 $. Also muss gelten $ z_1,z_2 \in \{-1,1\} $. Umgekehrt sind $ \{-1,1 \} $ zu sich selbst invers. 
}}

{{:Mathe für Nicht-Freaks: Vorlage:Aufgabe
 |titel=Produkte von Gruppen
 |aufgabe=Seien $ (G, \boxdot_G), (H, \boxdot_H) $ zwei Gruppen. Finde eine Verknüpfung $\odot:(G\times H)\times (G\times H) \rightarrow G\times H $ , mit der $ G \times H $ zu einer Gruppe wird.
 |lösung= 

Wir müssen zunächst eine innere Verknüpfung $ \odot: (G\times H) \times (G \times H) \to (G \times H) $ finden, dafür gibt es verschiedene Möglichkeiten, zum Beispiel die Folgende: 
Wir möchten ausnutzen, dass wir bereits zwei innere Verknüpfungen auf $ G$, bzw. $ H $ kennen, die eine Gruppenstruktur auf $ G $ bzw. $ H $ definieren. Durch komponentenweises Rechnen können wir so eine innere Verknüpfung auf $ G \times H $ definieren. Wir wählen also die Verknüpfung 
\begin{align*} \odot: (G\times H)\times (G\times H) \rightarrow G\times H\\[0.3em]
(a,c)\odot(b,d) = (a \boxdot_G b, c \boxdot_H d)
\end{align*}
Bei dieser Verknüpfung agieren die Elemente aus $ G, H $ nicht miteinander, die Komponenten verhalten sich also voneinander unabhängig. Die Gruppenstruktur von $ G $ und $ H $ bleibt dabei erhalten. Es gibt je nach Art der Gruppen auch viele andere Möglichkeiten, eine Gruppenstruktur auf dem kartesischen Produkt $ G \times H $ zu erzeugen. Diese sind vor allem dann interessant, wenn man möchte, dass die Elemente aus $ G $ und $ H $ miteinander agieren. Im Allgemeinen findet man keinen sinnvollen Weg, eine Abbildung zu definieren, die zwei Elemente verschiedener Gruppen zu einem neuen Element einer dieser Gruppen verknüpft. Die komponentenweise Verknüpfung ist besonders einfach, und außerdem für alle Paare von Gruppen anwendbar. 
Wir prüfen jetzt, dass $ \odot$ tatsächlich eine innere Verknüpfung ist. Dazu müssen wir Abgeschlossenheit zeigen.  Wir sehen, dass für alle $ a,b \in G $ und alle $ c,d \in H $ gilt
\begin{align*} (a,c) \odot(b,d) &=\\[0.3em]
&= (a\boxdot_G c, b\boxdot_H d) \in G\times H\end{align*}
Wir müssen jetzt nur noch zeigen, dass $ (G\times H, \boxdot) $ eine Gruppe ist, dazu rechnen wir alle Gruppeneigenschaften nach.
\# '''Assoziativität''': Wir sehen, dass für alle $ (a,d),(b,e)(c,f) \in G\times H$ gilt
\begin{align*} ((a,d) \odot (b,e)) \odot(c,f) &=\\[0.3em]
&= (a\boxdot_G b, d\boxdot_H e) \boxdot (c,f) \\[0.3em]
&= ((a\boxdot_G b) \boxdot_G c, (d \boxdot_H e) \boxdot_H f ) \\[0.3em]
&= (a\boxdot_G (b \boxdot_G c), d \boxdot_H (e \boxdot_H f)) \\[0.3em]
&= (a,c) \odot(b \boxdot_G c, e \boxdot_H f) \\[0.3em]
&= (a,d) \odot ((b,e) \odot(c,f))\\[0.3em]
\end{align*}
Um neutrales und inverse Elemente zu finden, gehen wir "komponentenweise" vor. 
\# '''Neutrales Element''': Wir wählen $ (e_1,e_2) $ als Kandidat für ein neutrales Element. Wir rechnen jetzt nach, dass $ (e_1,e_2) $ tatsächlich ein neutrales Element ist. Für alle $ (a,b) \in G\times H$ gilt:
\begin{align*} (a,b) \odot (e_1,e_2) &=\\[0.3em]
&= (a \boxdot_G e_1,b \boxdot_H e_2) \\[0.3em]
&= (a,b)
\end{align*}
und
\begin{align*} (e_1,e_2) \odot (a,b) &=\\[0.3em]
&= ( e_1 \boxdot_G a,e_2 \boxdot_H b) \\[0.3em]
&= (a,b)
\end{align*}
Somit ist $ (e_1,e_2) $ ein neutrales Element. 
\# '''Inverse Elemente''': Betrachten wir beliebiges $ (a,b) \in G\times H$, dann ist auch $ (a^{-1},b^{-1}) \in G\times H$. Insbesondere gilt:
\begin{align*} (a,b) \odot (a^{-1},b^{-1}) &=\\[0.3em]
&=(a \boxdot_G a^{-1},b \boxdot_H b^{-1}) \\[0.3em]
&=(e_1,e_2) \\[0.3em]
\end{align*}
und
\begin{align*} (a^{-1},b^{-1}) \odot (a,b) &=\\[0.3em]
&=(a^{-1} \boxdot_G a,b^{-1} \boxdot_H b) \\[0.3em]
&=(e_1,e_2) \\[0.3em]
\end{align*}
Somit ist $ (a^{-1},b^{-1}) $ das Inverse Element von $ (a,b) $ bezüglich $ \odot $.  
Wir haben jetzt gezeigt, dass$ (G\times H,\odot) $ eine Gruppe ist.
}}

Wir möchten unsere Konstruktion für das Produkt von Gruppen jetzt an einem expliziten Beispiel testen. 

{{:Mathe für Nicht-Freaks: Vorlage:Aufgabe
 |titel=$ \Z/12\Z \times \Z/60\Z $
 |aufgabe= Finde eine Operation, sodass $ \Z/12\Z \times \Z/60\Z $ mit dieser Operation zu einer Gruppe wird. Die so erzeugte Gruppenstruktur hat die folgende Interpretation: Denke an eine defekte Uhr, bei der Du Stunden- und Minutenzeiger separat bewegen kannst. Ein Element $ (a,b) \in \Z/12\Z \times \Z/60\Z$ kann man dabei als eine Verschiebung des Stundenzeigers um $ a $ Stunden bei gleichzeitiger  Verschiebung des Minutenzeigers um $ b $ Stunden auffassen. Hierbei sollen Minuten und Stundenzeiger unabhängig voneinander bewegt werden, das heißt, eine Verschiebung des Minutenzeigers um 60 Minuten führt NICHT zu einer Verschiebung des Stundenzeigers um eine Stunde. 
 |lösung= 
Wir haben bereits die Restklassengruppen $ \Z / n \Z $ kennengelernt. Man kann jede ganze Zahl mit einer Zahl aus $ \{0,...,n\} $ identifizieren, nämlich ihrem Rest bei Division durch $ n $. Weil diese Identifikation mit der Addition ganzer Zahlen verträglich ist (die Summe zweier Zahlen  ergibt bei Division durch $ n $ den gleichen Rest wie die Summe ihrer Reste ), vererbt die Addition ganzer Zahlen eine Gruppenstruktur auf $ \Z /n \Z $.  Einen Spezialfall davon haben wir bereits in der Einleitung kennengelernt: Das Verschieben des Stundenzeigers einer Uhr {{todo|verlinken}} ergibt eine Gruppenstruktur auf der Menge der möglichen Verschiebungen $ \{0,...,12\} $. Die dabei erzeugte Gruppe ist genau $ \Z / 12\Z $. 

In unserer Vorstellung sollen die Verschiebungen um Stunden und die Verschiebungen um Minuten unabhängig voneinander stattfinden. Das heißt, die zwei Komponenten des kartesischen Produkts sollen nicht miteinander agieren. Für eine Verknüpfung $ \boxplus: \Z/12\Z \times \Z/60\Z \to \Z/12\Z \times \Z/60\Z $ muss also gelten: $ ([a],[b]) \boxplus ([c],[d])= ([a] \boxplus_1[b], [c] \boxplus_2 [d]) $, für zwei Verknüpfungen $ \boxplus_1: \Z/12\Z \times \Z/12\Z \to \Z /12\Z $ und $\boxplus_2:  \Z/60\Z \times \Z/60\Z \to \Z /60\Z $.  Die Addition modulo $ 12 $ bzw. $ 60 $ eine Gruppenstruktur auf $ \Z/12 \Z $ zw. $ \Z / 60\Z $ definiert. Diese entspricht der Gruppe der möglichen Verschiebungen des Stunden- bzw. Minutenzeigers auf der Uhr (bezüglich Hintereinanderausführung). Wir werden jetzt versuchen, diese Verknüpfungen komponentenweise anzuwenden, um eine Gruppenstruktur auf dem kartesischen Produkt zu bauen. Wir definieren also:
\begin{align*} \boxplus : (\Z/12\Z \times \Z/60\Z) \times (\Z/12\Z \times \Z/60\Z)  \rightarrow \Z/12\Z \times \Z/60\Z \\[0.3em]
(([a],[b]), ([c],[d])) \rightarrow ([a+c], [b+d]) \end{align*}


Wir prüfen jetzt, ob $ \boxplus $ tatsächlich eine innere Verknüpfung ist. Dazu müssen wir Abgeschlossenheit zeigen.  Wir sehen, dass für alle $ ([a],[b])\in \Z/12\Z \times \Z /60\Z $ und alle $ ([c],[d]) \in \Z /12\Z \times \Z/60\Z $ gilt
\[([a],[b]) \boxplus ([c],[d]) = ([a+c], [b+d]) \in \Z/12\Z \times \Z/60\Z.\] Also ist $ \boxplus $ eine innere Verknüpfung. 
Wir rechnen jetzt die Gruppenaxiome für $ \boxplus $ nach.
\# '''Assoziativität''': Wir sehen,dass für alle $ a,b,c \in \Z/12\Z $ und alle $ d,e,f \in \Z/60\Z $ gilt
\begin{align*} (([a],[d]) \boxplus ([b],[e])) \boxplus ([c],[f])&=\\[0.3em]
&= ([a+b], [d+e]) \boxplus ([c],[f]) \\[0.3em]
&= (([(a+b)+ c], [(d+e) + f]) \\[0.3em]
&= ([a+(b + c)] , [d+(e + f)]) \\[0.3em]
&= ([a],[d]) \boxplus([b + c] , [e + f ])\\[0.3em]
&= ([a],[d]) \boxplus(([b],[e]) \boxplus ([c],[f]))\\[0.3em] 
\end{align*}
\# '''Neutrales Element''': Um ein neutrales Element zu finden, kehren wir nocheinmal zurück zu unserer Interpretation der Gruppenverknüpfung als separates Drehen der Stunden- und Minutenzeiger einer Uhr. Wir stellen fest: Die Bewegung des Stunden- und Minutenzeigers um jeweils Null Stunden bzw. Null Minuten ändert nichts, daher ergibt Vor- und Nachschalten dieser Bewegung an eine beliebige andere Verschiebung wieder diese Verschiebung. In Formeln heißt das: Für alle $ ([a],[b]) \in \Z/12\Z \times \Z/60\Z $ gilt:
\begin{align*} ([a],[b]) \boxplus ([0],[0]) &= \\[0.3em]
&= ([a] +[0] ,[b] + [0]) \\[0.3em]
&= ([a] , [b] ) \\[0.3em]
&= ([a],[b])
\end{align*}
und
\begin{align*} ([0],[0]) \boxplus ([a],[b]) &= \\[0.3em]
&= ([0] +[a] ,[0] + [b]) \\[0.3em]
&= ([a] , [b] ) \\[0.3em]
&= ([a],[b])  
\end{align*}
Deshalb ist $ ([0],[0]) $ das Neutrale Element unserer Gruppe.
\# '''Inverse Elemente''':  Wir werden jetzt versuchen, ein beliebiges, gegebenes Element $ ([a],[b]) $ "komponentenweise" zu invertieren. Betrachten wir $ ([a],[b]) \in \Z/12\Z \times \Z/60\Z $. Dann ist $ [12-a] $ invers zu $ [a] $ in $ \Z / 12 \Z $ und $ [60-b] $ ist invers zu $ [b] $ in $ \Z / 60 \Z $,  und es gilt:
\begin{align*} (a,b) \boxplus ([12-a] ,[60-b] ) &=\\[0.3em]
&=([a] + [12-a],[b] + [60-b] )\\[0.3em]
&=([12],[60]) \\[0.3em]
&=([0],[0]) \\[0.3em]
\end{align*}
und
\begin{align*} ([12-a] ,[60-b] )  \boxplus ([a],[b])&=\\[0.3em]
&=([12-a] + [a], [60-b] + [b])\\[0.3em]
&=([12],[60]) \\[0.3em]
&=([0],[0]) \\[0.3em]
\end{align*}
Somit ist $ ([12-a] ,[60-b] ) $ das Inverse Element von $ ([a],[b]) $ bezüglixh $ \boxplus $. 

Die Verknüpfung $ \boxplus $ macht also aus $ \Z/12\Z \times \Z/60\Z $ eine Gruppe. 


}}

{{:Mathe für Nicht-Freaks: Vorlage:Hinweis|
Es gibt auch andere Möglichkeiten, eine Gruppenstruktur auf $ \Z /12 \Z \times \Z / 60 \Z $ zu definieren. Die komponentenweise Definition ist aber besonders einfach. }}

\section{Die  symmetrische Gruppe $ S_n$}
Für $ n \in \N $ betrachte die Menge $ \{1,..., n\}$. Eine bijektive Abbildung $ \pi: \{1,..., n\} \to \{1,..., n\}  $ heißt Permutation. Die Permutationen sind also genau die 1:1- Zuordnungen von Elementen aus $ \{1,..., n\} $. Eine Permutation vertauscht die Zahlen aus $ \{1,...,n \} $ miteinander.
Für $ n=2 $ haben wir die Permutationen
\begin{align*} id: \{1,2\} &\to \{1,2\}\\ 1 &\mapsto 1 \\ 2& \mapsto 2 \end {align*} und \begin{align*}\tau: \{1,2\} &\to \{1,2\} \\1& \mapsto 2 \\ 2& \mapsto 1\end {align*}

{{:Mathe für Nicht-Freaks: Vorlage:Aufgabe
 |titel=$S_3$ 
 |aufgabe=Bestimme alle Permutationen für $ n=3 $ 
 |lösung=Es gibt sechs Permutationen über $ \{1,2,3\} $ . 
* \begin{align*} \text{id}: \{1,2,3\} &\to \{1,2,3\}\\1 &\mapsto 1 \\ 2& \mapsto 2 \\3& \mapsto 3 \end {align*}
* \begin{align*}\tau_{12}: \{1,2,3\} &\to \{1,2,3\}\\ 1 &\mapsto 2 \\ 2& \mapsto 1 \\3& \mapsto 3 \end {align*}
* \begin{align*}\tau_{13}: \{1,2,3\} &\to \{1,2,3\}\\1 &\mapsto 3 \\ 2 &\mapsto 2 \\3 &\mapsto 1 \end {align*} 
* \begin{align*} \tau_{23}: \{1,2,3\}& \to \{1,2,3\}\\ 1& \mapsto 1 \\ 2& \mapsto 3 \\3& \mapsto 2 \end {align*} 
* \begin{align*} \sigma_{123}: \{1,2,3\}& \to \{1,2,3\}\\ 1& \mapsto 2 \\ 2& \mapsto 3 \\3& \mapsto 1 \end {align*} 
* \begin{align*} \sigma_{132}: \{1,2,3\} &\to \{1,2,3\}\\1 &\mapsto 3 \\ 2 &\mapsto 1 \\3& \mapsto 2 \end {align*} 

Diese 6 Permutationen können wir mit den Umordnungen von Steinen aus der Einführung identifizieren: Wenn wir dem linken Platz die Nummer 1, dem mittleren Platz di Nummer 2 und dem rechten Platz die Nummer 3 zuweisen, dann geben diese Nummern an, welche Position auf welche (andere) Position verschoben wird. Die Gruppe, der Vertauschungen von Steinen ist also genau $ S_2 $.  
}}
Wenn man zwei Permutationen miteinander verknüpft, also hintereinanderausführt, erhält man wieder eine Permutation. (Die Komposition zweier Permutationen $ \pi_1, \pi_2: \{1,...,n\} \to \{1,..,n\}$ zu $\pi_2 \circ\pi_1: \{1,..,n\} \to \{1,...,n \}; x \mapsto \pi_2 (\pi_1 (x)$)  ist eine Abbildung von $ \{1,...,n\} $ nach $ \{1,...,n\} $, und bijektiv als Verkettung bijektiver Abbildungen. Etwas intuitiver ausgedrückt: Jede Vertauschung von Elementen aus $ \{1,...,n\} $, die in mehreren Runden durchgeführt wird (also durch Hintereinanderausführung von mehreren Vertauschungen/Permutationen), ist auch in einem Durchgang realisierbar. 


Mache Dir klar, wieso die Verkettung zweier bijektiver Abbildung wieder bijektiv ist! 
Wie wir gleich beweisen werden, bildet die Menge der Permutationen über $\{1,...,n\} $ mit der Komposition eine Gruppe. Sie heißt "symmetrische Gruppe" in $n$ Elementen. Die symmetrische Gruppe ist nicht kommutativ für $ n > 2 $. Wir bezeichnen die Menge der Permutationen über $\{1,...,n\}$ von nun an mit $ S_n $

*Die Verkettung $ \pi_2 \circ \pi_1 $ von zwei Permutationen $ \pi_1 , \pi_2 : \{1,...,n\} \to \{1,...,n \}$ ist eine bijektive Abbildung $ \{1,...,n \} \to \{1,...,n \} $, also eine Permutation. Daher ist die Verknüpfung $ \circ : S_n \times S_n \to S_n (\pi_1,\pi_2) \mapsto \pi_1 \circ \pi_2 $, die zwei Permutationen miteinander verkettet, wohldefiniert.
*$ \circ $ ist assoziativ

{{:Mathe für Nicht-Freaks: Vorlage:Beweis
 |titel=Assoziativität der Verknüpfung in der symmetrischen Gruppe
 |beweis=
Bevor du damit beginnst den Beweis zu lesen, ist es sinnvoll sich die Definition der [[Mathe für Nicht-Freaks: Verknüpfung\#Assoziativität|Assoziativität]] nachzuschlagen. 

Wir fixieren zunächst drei beliebige Permutationen $\pi_1, \pi_2, \pi_3 : \{1,...,n\} \to \{1,...,n\}$. Wir wollen zeigen, dass 
\[\pi_1 \circ ( \pi_2 \circ \pi_3 ) = ( \pi_1 \circ \pi_2 ) \circ \pi_3\]
gilt. Da auf der linken bzw. der rechten Seite jeweils Funktionen stehen, können wir diese Gleichung dadurch zeigen, dass wir auf jedes Element $m\in\{1,...,n\}$ die linke und die rechte Seite anwenden und nachweisen, dass das gleiche herauskommt. 

Sei also $m\in\{1,...,n\}$ beliebig. Beschäftigen wir uns zuerst mit der linken Seite. Hier setzen wir $m$ in 
\[\pi_1 \circ ( \pi_2 \circ \pi_3 )\]
ein. Dies ist eine Komposition der Funktionen $\pi_1$ und $\pi_2 \circ \pi_3$. Das heißt wir berechnen
\[( \pi_1 \circ ( \pi_2 \circ \pi_3))(m)\]
dadurch, dass wir das Ergebnis von $(\pi_2 \circ \pi_3)(m)$ nehmen und in $\pi_1$ einsetzen. Soweit so gut. Nun bleibt nur zu klären, was $(\pi_2 \circ \pi_3)(m)$ ist. Dazu verfahren wir auf die gleiche Art und Weise:
Es ist diejenige Zahl, die herauskommt, wenn man $\pi_3(m)$ in $\pi_2$ einsetzt, also
\[\pi_2 ( \pi_3 ( m ) ).\]
Puzzeln wir das nun mit dem ersten Schritt zusammen, so ergibt sich als Ergebnis von $(\pi_1 \circ ( \pi_2 \circ \pi_3 ))(m)$:
\[\pi_1( \pi_2 ( \pi_3 ( m ) ) ).\]

Nun kümmern wir uns um die rechte Seite. Das heißt, wir wollen 
\[((\pi_1 \circ \pi_2 ) \circ \pi_3)(m)\]
ohne das Verknüpfungssymbol $\circ$ schreiben (genau wie oben für die linke Seite). Zunächst kümmern wir uns um den äußeren (rechten) Kringel. Er ist deswegen der "äußere", weil der anders als der linke Kringel innerhalb weniger Klammern steht. Gehen wir nun nach diesem Prinzip vor, so berechnet sich 
\[((\pi_1 \circ \pi_2 ) \circ \pi_3)(m)\]
dadurch, dass man $\pi_3(m)$ in $\pi_1 \circ \pi_2$ einsetzt. 

Nun müssen wir klären was passiert, wenn man irgendeine Zahl $m'$ in $\pi_1 \circ \pi_2$ einsetzt. Weiter unten wollen wir $m'= \pi_3(m)$ setzen. Wie oben ergibt sich 
\[(\pi_1\circ \pi_2)(m')\]
daraus, dass wir $\pi_2(m')$ in $\pi_1$ einsetzen. 

Damit erhalten wir
\[((\pi_1 \circ \pi_2)\circ \pi_3)(m) = \pi_1( \pi_2 ( \pi_3 ( m ) ) ).\]
Das trifft sich sehr gut, da wir das schon für die linke Seite erhalten haben. Also haben wir für alle $m\in\{1,...,n\}$ die Gleichung
\[(\pi_1 \circ ( \pi_2 \circ \pi_3))(m) = \pi_1(\pi_2(\pi_3(m))) = (( \pi_1 \circ \pi_2 ) \circ \pi_3)(m)\]
nachgewiesen, woraus die Gleichheit von Funktionen
\[\pi_1 \circ ( \pi_2 \circ \pi_3) = (\pi_1 \circ \pi_2) \circ \pi_3\]
folgt. 
}}
{{:Mathe für Nicht-Freaks: Vorlage:Hinweis| Allgemein gilt: Sei M eine nichtleere Menge. Die Verkettung von Abbildungen $M \to M$ ist eine assoziative Verknüpfung auf der Menge der Selbstabbildungen von M. Der Beweis dieser allgemeineren Aussage funktioniert analog und ist eine gute Übung. }}


*Die Identität $ id: \{1,...,n\} \to \{1,...,n \}; x \mapsto x $ ist das neutrale Element, sie vertauscht keine Elemente. 
*Eine Permutation ist eine bijektive Abbildung von $ \{1,...,n\} $ nach $ \{1,...,n\}$. Jede Permutation $ \pi $ hat daher eine Umkehrabbildung $ \pi^{-1} : \{1,...,n\} \to \{1,...,n \} $, welche die Zahlen $ 1,...,n $ auf ihre Urbilder (unter $ \pi $ zurückschickt, es gilt $ \pi^{-1} \circ \pi = id $. Weil $ \pi^{-1} $ eine Abbildung von $ \{1,...,n \} $ nach $ \{1,...,n \} $ bijektiv ist (mit Umkehrabbildung $ \pi$), ist $ \pi^{-1} $  ebenfalls eine Permutation aus $ S_n $ und somit ein inverses Element zu $ \pi $. 

$ S_n $ bildet also mit der Verknüpfung $ \circ $ eine Gruppe. Man nennt $(S_n,\circ) $ die n-te symmetrische Gruppe. 

Für $ n > 2 $ ist $ S_n $ nicht abelsch, denn: 
Für $ n >2 $ sind \begin{align*} \tau_{12}: \{1,...,n\} &\to \{1,...,n\} \\ 1& \mapsto 2\\ 2& \mapsto 1 \\ x& \mapsto x \text{ falls } x>2  \end{align*}

und \begin{align*}  \tau_{23}: \{1,...,n\} &\to \{1,...,n \} \\ 1& \mapsto 1\\ 2& \mapsto 3 \\ 3& \mapsto 2 \\ x &\mapsto x \text{ falls }x >3 \end{align*} 
Permutationen aus $ S_n$. Es gilt $ \tau_{23} \circ \tau_{12} (1)= \tau_{23} (2)= 3$, aber $ \tau_{12} \circ \tau_{23} (1) = \tau_{12} (1) =2 $, also ist $ \tau_{23}\circ \tau_{12} \neq \tau_{12} \circ \tau_{23} $. Die Verknüpfung $ \circ $ ist daher nicht kommutativ, also ist $S_n$ keine Abelsche Gruppe (für $ n > 2 $).  


Wenn Du noch mehr über die n-te symmetrische Gruppe lernen möchtest, schau Dir doch den \ref{Gruppen:SatzCayley}[[\#Anker:SatzCaley|Satz von Cayley]] am Ende dieses Artikels an.

\section{Untergruppen}
\label{Gruppen:Untergruppen}
Eine grundlegende Eigenschaft einer Gruppe ist ihre Abgeschlossenheit unter der Gruppenoperation. Bezogen auf die tabellarische Darstellung einer Gruppe, wie wir sie etwa in der Einleitung für die Vertauschungen dreier Steine vorgenommen haben, bedeutet dies, dass im Ergebnisbereich exakt dieselben Objekte vorkommen, wie in der ersten Zeile und Spalte.
Tatsächlich kann es jedoch vorkommen, dass bereits ein Teil einer solchen Tabelle diese Forderung erfüllt, ohne dass wir den Rest der Tabelle mit einbeziehen müssen. Als Beispiel betrachten wir die ersten zwei Zeilen und Spalten des Ergebnisbereichs der Tabelle der Steinvertauschungen. Wenn wir diese isolieren, erhalten wir folgende Teiltabelle:
[[File:S3 Cayley table S2 subgroup.svg|frameless|center|upright=1.5||]]
Sowohl in der Kopfzeile, als auch in der ersten Spalte dieser Tabelle erscheinen lediglich zwei der sechs Vertauschungen (die Identitätstransformation und das Vertauschen des mittleren mit dem rechten Stein). Die Tabelle betrachtet also alle denkbaren Verknüpfungen dieser beiden Gruppenelemente. Im Ergebnisbereich erkennen wir, dass auch hier nur diese beiden speziellen Vertauschungen auftreten. Die Teiltabelle ist also für sich bereits abgeschlossen.

Wir können diese Beobachtung auch ohne das Hilfsmittel der Tabelle formulieren. Wir sagen: Die Teilmenge $ \{ \text{Identitätstransformation}, \text{Mitte-Rechts-Vertauschung} \} $ der Menge aller Vertauschungen dreier Steine ist abgeschlossen unter der Gruppenoperation (Verknüpfung von Vertauschungen).

Weitere Beispiele für Teilmengen von Gruppen, die für sich bereits unter der Gruppenoperation abgeschlossen sind, umfassen:
Die geraden Zahlen innerhalb der ganzen Zahlen, die Viertelstunden auf der Uhr (also :00, :15, :30 und :45), Addition und Multiplikation der rationalen Zahlen als Teilmenge der reellen Zahlen, und viele mehr.

Da wir uns in diesem Artikel aber nicht nur mit abgeschlossenen Operationen, sondern mit Gruppen beschäftigen, liegt es nahe zu fragen, ob diese Teilmengen auch die anderen Eigenschaften erfüllen, die wir von Gruppen fordern. Um diese Frage auf den Punkt zu bringen, definieren wir das Konzept der "Untergruppe":

{{:Mathe für Nicht-Freaks: Vorlage:Definition
 |titel=Untergruppe
 |definition=
 Sei $(G, \boxdot)$ eine Gruppe. Eine Teilmenge $U \subset G$ heißt '''Untergruppe (von G)''', falls
\#$U \neq \emptyset$
\#für alle $u, v \in U$ gilt $u \boxdot v \in U$. Man sagt auch U ist '''abgeschlossen''' unter $\boxdot$. 
\#für alle $ u \in U$ gilt $u^{-1} \in U$
\#das neutrale Element $e_{G}$ aus $G$ liegt in $U$. Dies folgt bereits aus den ersten 3 Punkten (Übungsaufgabe, siehe unten)
}}

{{:Mathe für Nicht-Freaks: Vorlage:Hinweis|
Eine Teilmenge $U \subset G$ einer Gruppe $(G, \boxdot)$, die eine Untergruppe ist, bildet selbst wieder eine Gruppe $(U, \boxdot_{U})$ mit der Verknüpfung $\boxdot_{U} : U \times U \to U, (v,w) \mapsto v \boxdot w$. Der Einfachheit halber schreibt man anstatt $\boxdot_{U}$ oft nur $\boxdot$. Außerdem benutzt man die Schreibweise $U \subset G$ in Anlehnung an die Mengeninklusion auch für die ganzen Gruppen. Man liest also "U ist eine Untergruppe von G".
}}

Die Assoziativität der Verknüpfung der Untergruppe folgt direkt aus der Assoziativität der ursprünglichen Gruppe. Außerdem besitzt die Untergruppe dasselbe neutrale Element, und Elemente, die in der Untergruppe invers zueinander sind, sind dies ebenfalls in der ursprünglichen Gruppe. Deshalb sagt man oft, die Untergruppe "erbt" ihre Eigenschaften.

\subsection{Wie überprüft man, ob eine Teilmenge eine Untergruppe ist?}
{{:Mathe für Nicht-Freaks: Vorlage:Satz
 |satz= 
Sei $(G, \boxdot)$ eine Gruppe.
Eine Teilmenge $U \subset G$ ist eine Untergruppe, genau dann wenn gilt:
\#$U \neq \emptyset$
\#Für alle $u, v \in U$ gilt $u \boxdot v^{-1} \in U$
 |beweis=
{{:Mathe für Nicht-Freaks: Vorlage:Beweisschritt
 |ziel="$\implies$" 
 |beweisschritt=$U \neq \emptyset$ folgt direkt aus der Definition der Untergruppe.
Seien $u, v \in U$. Da $U$ Untergruppe gilt $v^{-1} \in U$, und damit ist $u \boxdot v^{-1} \in U$
}}
{{:Mathe für Nicht-Freaks: Vorlage:Beweisschritt
 |ziel="$\Longleftarrow$"
 |beweisschritt=$U \neq \emptyset$ folgt aus Eigenschaft 1.
Zunächst zeigen wir, dass das neutrale Element $e \in G$ in $U$ liegt.
Da $U \neq \emptyset$, existiert ein $u \in U$.
Eigenschaft 2 impliziert dann, mit $v=u$, dass $e = u \boxdot u^{-1} \in U$.
Nun zeigen wir, dass für ein $u \in U$ auch gilt $u^{-1} \in U$:
Aus Eigenschaft 2 mit $v = e \in U$ folgt: $u^{-1} = e \boxdot u^{-1} \in U$
Zum Schluss zeigen wir noch, dass für $u, v \in U$ gilt: $u \boxdot v \in U$
Aus dem vorherigen Schritt wissen wir bereits, dass $v^{-1} \in U$. 
Dann folgt aber wieder aus Eigenschaft 2, dass $u \boxdot v = u \boxdot (v^{-1})^{-1} \in U$
}}
}}

{{:Mathe für Nicht-Freaks: Vorlage:Aufgabe
 |aufgabe=Sei $G$ eine Gruppe, $U \subset G$ eine Teilmenge, die Eigenschaften 1-3 aus der Definition der Gruppe erfüllt. Es gelte also 
\#$U \neq \emptyset$
\#für alle $u, v \in U$ gilt $u \boxdot v \in U$. Man sagt auch U ist '''abgeschlossen''' unter $\boxdot$. 
\#für alle $ u \in U$ gilt $u^{-1} \in U$
Zeige, dass $e_G \in U$
 |beweis=Da $U \neq \emptyset$, gibt es ein $u \in U$.
Da $U$ abgeschlossen unter Inversenbildung ist, folgt $u^{-1} \in U$.
Da $U$ abgeschlossen unter Multiplikation ist, gilt $e_G = u\boxdot u^{-1} \in U$
}}

\subsection{Beispiele}
{{:Mathe für Nicht-Freaks: Vorlage:Beispiel
 |titel=Triviale Untergruppe
 |beispiel=
Sei $G$ eine Gruppe. Dann ist $\{e_{G}\} \subset G$ eine Untergruppe von $ G $
Man nennt diese Untergruppe die '''triviale Untergruppe''', da sie immer existiert.
}}
{{:Mathe für Nicht-Freaks: Vorlage:Beispiel
 |titel=
 |beispiel=
Sei $G$ eine Gruppe. Dann ist $G\subset G$ eine Untergruppe von sich selbst. 
}}
{{:Mathe für Nicht-Freaks: Vorlage:Beispiel
 |titel=$\Z$ als Untergruppe von $\Q, \R, \C$. 
 |beispiel=Wir haben schon gesehen, dass die ganzen Zahlen $ \Z $ mit der gewöhnlichen Addition eine Gruppe bilden.
 Ebenso sind $ \Q, \R, \C $, ausgestattet mit der Addition, Gruppen. Wir können die ganzen Zahlen als Teilmenge dieser Zahlbereiche auffassen. Weil die ganzen Zahlen unter der Einschränkung der Addition (vom größeren Zahlbereich $ \Q, \R, \C$ ) auf die ganzen Zahlen selbst eine Gruppe bilden, ist $ (\Z,+) $ Untergruppe von $ (\Q,+),(\R,+)$ und $ (\Complex, +) $. 
}} 

Allgemeiner gilt das Folgende.
{{:Mathe für Nicht-Freaks: Vorlage:Beispiel
 |titel=Untergruppen von Untergruppen
 |beispiel=Sei $ (G, \boxdot) $ eine Gruppe, und $ U \subset G $ eine Untergruppe.
Dann ist $ U $ mit der Verknüpfung $ \boxdot_U: U\times U \to U, (u,v) \mapsto u \boxdot  v $ selbst eine Gruppe. Sei $ V \subset U $ eine Untergruppe. Dann ist $ V $ auch Untergruppe von $G $. 

Tatsächlich: Da $ V \subset U $ eine Untergruppe ist, ist $ V $ mit der Verknüpfung $ \boxdot _V : V \times V \to V, (v,w) \mapsto v \boxdot_U w= v\boxdot w $ eine Gruppe. Weil zudem $ V $ eine nichtleere Teilmenge von $U$, und somit auch von $G$ ist, muss $V$ eine Untergruppe von $G$ sein. 
}}

{{:Mathe für Nicht-Freaks: Vorlage:Beispiel
 |titel=$2\Z$ als Untergruppe von $(\Z,+)$
 |beispiel=
 Die geraden ganzen Zahlen  $ 2 \Z $ (das sind genau die Zahlen $z$ der Form $ z=2 \cdot k, k \in  \Z $) sind eine Untergruppe von $ (\Z, + )$. $ + $  bezeichnet dabei die gewöhnliche Addition ganzer Zahlen. 
}}
{{:Mathe für Nicht-Freaks: Vorlage:Beweis
 |beweis=
Wir zeigen, dass $2\Z \subset \Z$ eine Untergruppe ist (bezüglich der Addition). Dazu verwenden wir den obigen Satz. 
Da $0 = 2 \cdot 0 \in 2\Z$, ist  $2\Z \neq \emptyset$.
Für $ b \in \Z $ ist $ -b $ das additive Inverse. Nach dem obigen Satz müssen wir nur noch zeigen, dass falls $a, b \in 2\Z$, dann ist auch $a - b \in 2\Z$.
Seien dazu $a,b\in 2\Z$. Das bedeutet, es gibt $m,n \in \Z$ mit $a = 2\cdot m, b=2\cdot n$
Dann ist aber auch $a - b = 2\cdot m - 2 \cdot n = 2\cdot(m-n) \in 2\Z$.

Also ist $2\Z \subset \Z$ eine Untergruppe.
}}

Dieses Beispiel gibt Anlass zu folgender Verallgemeinerung: 

{{:Mathe für Nicht-Freaks: Vorlage:Satz
 |titel=Untergruppen von $(\Z,+)$
 |satz=Sei $U \subset \Z$. Dann gilt folgende Äquivalenz:
 
$U$ ist Untergruppe $\iff$ Es gibt eine ganze Zahl $n \in \Z$, sodass $U = n\Z = \lbrace n \cdot z \big| z \in \Z \rbrace$

{{:Mathe für Nicht-Freaks: Vorlage:Beweisschritt
 |ziel="$\Leftarrow$
 |beweisschritt=Sei $n \in \Z$, sodass $U = n\Z$. 
Es ist $U \neq \emptyset$, da $0 \in U$.

Seien $u_1, u_2 \in U$. Dann gibt es $v_1, v_2 \in \Z$, sodass gilt $u_1 = n \cdot v_1, u_2 = n \cdot v_2$

Dann ist auch $u_1 + (-u_2) = nv_1 - nv_2 = n(v_1 - v_2) \in  n\Z$

Also ist $U \subset \Z$ Untergruppe. Dabei haben wir den Satz über äquivalente Charakterisierungen von Untergruppen verwendet. 
}}

{{:Mathe für Nicht-Freaks: Vorlage:Beweisschritt
 |ziel="$\implies$"
 |beweisschritt=Sei $ U \subset Z $ eine Untergruppe. Wenn $U=\{0\}$ die triviale Untergruppe ist, dann ist $U=\{0\}=0 \Z$ und die Behauptung stimmt.

Wenn $U$ nicht die triviale Untergruppe ist, gibt es ein $u\in U$ mit $u\neq 0$. Mit $ u$ ist auch das Inverse $ -u\in U$, und eine der beiden Zahlen muss positiv sein. Es gibt also positive Elemente in $U$.

Sei $ n:= min \{ u \in U | u > 0 \}$ die kleinste positive Zahl aus $ U $. Wir werden zeigen, dass $ U = \{n*z| z \in Z \} $. 
Zuerst zeigen wir $ \subseteq  $. 
Sei $ u \in U $ ein beliebiges Element. Dann können wir $ u $ mit Rest durch $n $ teilen,
und erhalten $ u=q \cdot n+r $ mit  $ q,r \in Z $ und $ 0 \leq r \leq n-1 $.
Weil $ q \cdot n= \underbrace{n+n+...+n}_{ q mal} \in U $ (falls q >0),
$ q \cdot n= \underbrace {-n+(-n)+...+(-n)}_{|q| mal} \in U $ und $ 0*n=0 \in U $ gilt in jedem Fall $ q \cdot n \in U$. Deshalb ist auch $ r= u-q \cdot n \in U $. 
  
Weil $ n $ die kleinste positive Zahl aus $ U $ ist, und $ r <n $, ist $ r $ nicht positiv, also $ r \leq 0 $.Weil wir angenommen hatten, dass $ 0 \leq r $ muss also $ r=0 $ gelten. Insgesamt ergibt sich $ u= q \cdot n \in \{ n \cdot z | z \in Z \} $. Weil $ u $ ein beliebiges Element aus $ U $ war, folgt $ U \subseteq \{ n \cdot z | z \in Z \} $. 
  
Jetzt müssen wir noch $ \supseteq $ zeigen.
Es gilt $ n \in U $, und da $ U $ eine Gruppe ist, gilt für $ z \in Z $, dass $ n \cdot z= \underbrace{ n+n+...+n}_{z mal} \in U $ (falls z >0), und $ n \cdot z= \underbrace { (-n)+(-n)+..+(-)n}_{|z| mal} \in U $ (falls $ z<0$). Falls z=0, gilt $ n \cdot z=0 \in U $. Also liegt für alle $ z \in \Z n \cdot z $ in $ U $. Deshalb gilt $ \{n \cdot z | z \in Z \} \subset U $.  
}}
}}

{{:Mathe für Nicht-Freaks: Vorlage:Hinweis
 |Der obige Beweis zeigt auch, dass $\Z$ ein Hauptidealring ist. Falls Du noch nicht weißt, was ein Hauptidealring ist, musst Du Dir an dieser Stelle keine Gedanken darüber machen. 
}}

\section{Halbgruppen und Monoide}
Nachdem wir nun die Definition der Gruppe etwas kennengelernt haben, ergibt sich vielleicht die Frage: Gibt es algebraische Objekte mit weniger Struktur als Gruppen, aber mehr Struktur als Mengen?

Die Antwort ist ja, und sie haben sogar Namen:
Da diese Objekte im Vergleich zu Gruppen nur eine geringe Rolle spielen, werden wir uns hier auf ihre Definitionen beschränken. 

Eine Halbgruppe $ (H , \boxdot )$ ist eine Menge mit einer assoziativen Verknüpfung $ \boxdot : H \times H \to H $. Abgesehen von der Assoziativität muss die Menge unter der Verknüpfung keine weiteren Eigenschaften erfüllen. 
Eine Halbgruppe erfüllt also genau die ersten beiden Eigenschaften aus der Definition der Gruppe. 
Im Allgemeinen haben Halbgruppen weder inverse noch neutrale Elemente, aber natürlich gibt es Halbgruppen die ein neutrales und/oder ein inverses Element enthalten. Insbesondere ist jede Gruppe auch eine Halbgruppe (unter der Gruppenverknüpfung). 
Noch eine weitere Besonderheit von Halbgruppen: Es existieren Halbgruppen mit linksneutralem oder rechtsneutralem Element, aber ohne neutrales Element. 
{{todo|Beispiel für Gruppe ohne neutralem aber mit rechts-/linksneutralem Element einfügen}}

Eine Halbgruppe, die ein neutrales Element enthält, heißt Monoid. Monoide erfüllen die ersten drei Axiome aus der Definition der Gruppe, ihnen fehlt "nur" die Existenz von Inversen.

{{:Mathe für Nicht-Freaks: Vorlage:Beispiel
 |titel=Halbgruppen und Monoide
 |beispiel=
 
Ein Beispiel für eine Halbgruppe bilden die natürlichen Zahlen mit der Addition, also $(\N,+) $. Das Assoziativitätsgesetz gilt ja in $ \N$.

Da die Gleichungen $ 1+x=1$ und $x+1=1$ in $\N$ unlösbar sind, bildet $(\N,+) $ kein Monoid.

Nehmen wir als Verknüpfung die Multiplikation, so erhalten wir ein Monoid. Die Assoziativität der Multiplikation gilt in $ (\N,\cdot)$, und  die $ 1$ ist das neutrale Element. Weil die Gleichung $ 2\cdot x=1$ in $ \N$ aber unlösbar ist, hat $ 2\in\N$ kein Inverses, und wir haben keine Gruppe.

Weitere Beispiele sind $ (2\Z,\cdot)$ (Halbgruppe) und $ (\Z,\cdot)$ (Monoid).
}}

\section{Satz von Cayley}
\label{Gruppen:SatzCayley}

Der folgende Satz zeigt, dass jede endliche Gruppe Untergruppe einer symmetrischen Gruppe ist.
Zum Beweis des Satzes benötigen wir das Konzept von strukturerhaltenden Abbildungen zwischen Gruppen (also Gruppenhomomorphismen).
Falls du noch nicht weißt, was das ist, kannst du diesen Satz ohne Probleme überspringen, er befindet sich aus konzeptionellen Gründen hier.

{{:Mathe für Nicht-Freaks: Vorlage:Satz
 |titel=Satz von Cayley
 |satz=Jede Gruppe mit $n$ Elementen ist isomorph zu einer Untergruppe der Gruppe $S_n$.
 |beweis=Sei $G$ eine Gruppe mit $n$ Elementen.
Sei $g \in G$ ein Element. Betrachte die Abbildung (von Mengen, nicht von Gruppen)
$\mu_g: G \to G, h \mapsto gh$.

{{:Mathe für Nicht-Freaks: Vorlage:Beweisschritt
 |ziel=$\mu_g$ ist eine Bijektion.
 |beweisschritt=Da $G$ eine endliche Menge ist, genügt es zu zeigen, dass $\mu_g$ injektiv ist.
Seien dazu $h_1, h_2 \in G$ mit $\mu_g(h_1) = g \boxdot h_1=\mu_g(h_2)= g \boxdot h_2 $.
Da $G$ eine Gruppe ist, existiert ein Inverses zu $g$, und es folgt $ h_1 = g^{-1}\boxdot(g\boxdot h_1) = g^{-1}\boxdot(g\boxdot h_2) = h_2$, wobei wir die Assoziativität von $ \boxdot $ verwendet haben.
Also ist $\mu_g$ injektiv, und damit bijektiv.
}}

Da $G$ endlich ist, und genau $ n $ Elemente besitzt, können wir die Elemente von G durchnummerieren. Dabei ordnen wir jedem Element $ g  \in G $ genau eine Nummer aus $\{1,..., n\} $ zu, oder formaler ausgedrückt: Wir finden eine Bijektion $\iota: G \to \lbrace 1, ..., n \rbrace$ 

Wir wollen nun einen injektiven Gruppenhomomorphismus $F: G \to S_n$ konstruieren. (Dann können wir $G$ als Untergruppe von $S_n$ auffassen). Ein Gruppenhomomorphismus ist eine Abbildung zwischen zwei Gruppen, die mit der Gruppenstruktur verträglich ist, das heißt: Seien $ (G, \boxdot), (H, \tilde{\boxdot}) $ zwei Gruppen. Eine Abbildung $\Phi: G \to H $ ist ein Gruppenhomomorphismus, falls für alle $ g,f \in G $ gilt: $ \Phi(g \boxdot f) = \Phi(g) \tilde{ \boxdot} \Phi(f) $ und zusätzlich $ \Phi(e_G)=e_H $.

Dazu definieren wir für ein $g \in G$:

\begin{align*} F(g) : \lbrace 1, ..., n \rbrace &\to \lbrace 1, ..., n \rbrace \\ i &\mapsto \iota(\mu_{g}(\iota^{-1}(i))) \end{align*}

Wir sehen, dass $F(g) = \iota \circ \mu_g \circ \iota^{-1}$ als Komposition von Bijektionen wieder eine Bijektion ist, also liegt $F(g) \in S_n$ für alle $g \in G$. Wir haben somit eine Abbildung $ F: G \to S_n , g \mapsto F(g) $ gefunden. 
Wir müssen noch zeigen, dass $F$ mit der Gruppenstruktur verträglich (also ein Gruppenhomomorphismus) und injektiv ist.

{{:Mathe für Nicht-Freaks: Vorlage:Beweisschritt
 |ziel=$J(e_G) = id$
 |beweisschritt=Sei $i \in \lbrace 1, ..., n \rbrace$. Dann gilt: 
 
$F(e_G)(i) = \iota(\mu_{e_G}(\iota^{-1}(i))) = \iota(e_G \cdot \iota^{-1}(i)) = \iota(\iota^{-1}(i)) = i$
Also ist $F(e_G) = id$ die Identitätsabbildung. 
}}

{{:Mathe für Nicht-Freaks: Vorlage:Beweisschritt
 |ziel=$F(g)\circ F(h) = F(gh)$ für alle $g, h \in G$
 |beweisschritt=Seien $g,h \in G, i \in \lbrace 1, ..., n \rbrace$. Dann gilt: 
 
$(F(g) \circ F(h))(i) = ((\iota \circ \mu_g \circ \iota^{-1}) \circ (\iota \circ \mu_h \circ \iota^{-1}))(i) = (\iota \circ \mu_g \circ \mu_h \circ \iota^{-1})(i) = \iota(gh \cdot \iota^{-1}(i)) = \iota(\mu_{gh}(\iota^{-1}(i))) = F(gh)(i)$

Also $F(g) \circ F(h) = F(gh)$.
}}
Wir haben somit gezeigt, dass $ F $ ein Gruppenhomomorphismus ist.
{{:Mathe für Nicht-Freaks: Vorlage:Beweisschritt
 |ziel=$F$ ist injektiv.
 |beweisschritt=
Seien $ g,h \in G $ mit $ F(g)=F(h) $. Wir wollen zeigen, dass $ g=h $. 

Da $ F $ ein Gruppenhomomorphismus ist, ist $F(h) \circ F(h^{-1})= F(h h^{-1})=F(e_G)= id $. Also ist $ F(h^{-1}) $ das inverse Element von $ F(h)=F(g) $. Es gilt deshalb $ id=F(g) \circ F(h^{-1})=F(gh^{-1}) $. 

Die Menge $ \operatorname{ker}(F):= \{g \in G | F(g)= id \} $ nennt man Kern von $ F $. Wenn wir zeigen können, dass $ Ker(F)= \{e_G\} $, so folgt daraus (weil $ gh^{-1} \in \operatorname{ker} (F) $), dass $ g h^{-1}= e_G $, also $ g=h $. Es reicht also, $ \operatorname{ker}(F)= \{e_G \} $ zu zeigen, dann folgt bereits die Injektivität von $ F $.

Für alle Gruppenhomomorphismen $ \Phi : G \to H $ gilt wegen $ \Phi(e_G)=e_H $, dass $ e_G \in \operatorname{ker} ( \Phi) $.  

Sei $g \in \operatorname{ker}(F)$. Das heißt, $\iota(g \cdot \iota^{-1}(i)) = i$ für alle $1 \leq i \leq n$. Wenn wir auf beide Seiten der Gleichung $ \iota^{-1} $ anwenden, folgt daraus bereits $g \cdot \iota^{-1}(i) = \iota^{-1}(i), 1 \leq i \leq n$. Weil $ \iota^{-1} $ bijektiv (insbesondere also surjektiv) ist, finden wir für $ g \in G $ ein $ i \in \{1,...,n \} $, sodass $\iota^{-1}(i)=g $. Es gilt somit für alle $ h \in G $, dass $ g \cdot h = h $. $ g $ muss also das neutrale Element sein. 
Daher ist $g = e_G$, wie gewünscht, und somit $ \operatorname{ker}(F)= \{e_G \} $, also $ F $ injektiv.
}}
Weil $ F : G \to S_n $ injektiv ist, ist $ G $ isomorph zu seinem Bild unter $F $, welches eine Untergruppe von $ S_n $ ist. 
}}


{{:Mathe für Nicht-Freaks: Vorlage:Hinweis
 |Im Beweis sehen wir, dass die Einbettung $F: G \to S_n$ von der Wahl einer Bijektion $\iota: G \to \lbrace 1, ..., n \rbrace$ abhängt.
 
$G$ taucht also öfter als Untergruppe von $S_n$ auf
}}

\section{Exkurs: Abschwächung der Gruppenaxiome}

Man kann die in unserer Definition geforderten Eigenschaften einer Gruppe etwas abschwächen, und zeigen, dass Mengen (mit inneren Verknüpfungen), die diese abgeschwächten Eigenschaften haben, bereits alle von uns geforderten Gruppeneigenschaften erfüllen. Da unsere Definition etwas leichter zu merken ist, und sich aus diesen Abschwächungen in der Anwendung nur geringe Vorteile ergeben, haben wir hier darauf verzichtet. Der Vollständigkeit halber, und um Verwirrung, die sich bei der Hinzunahme anderer Quellen ergeben könnte, zu vermeiden, möchten wir dennoch kurz darauf eingehen. Wir werden im Folgenden aber stets mit "unserer" Definition der Gruppe arbeiten, weshalb Du den folgenden Abschnitt problemlos überspringen kannst.


Sei $ \emptyset  \neq G $ eine Menge, und $ \boxdot $ eine innere Verknüpfung auf $ G $. Ein Element $ e_G \in G $ heißt rechtsneutral (bezüglich $ \boxdot $), falls für alle $m \in G $ gilt $ m \boxdot e_G =m $, wenn es also von rechts wie ein neutrales Element wirkt. Umgekehrt heißt ein Element linksneutral, falls $ e_M \boxdot m=m $ für alle $ m \in G $. Neutrale Elemente sind Elemente, die gleichzeitig rechts- und linksneutral sind. 

Analog dazu kann man (vorausgesetzt, dass rechts- bzw. linksinverse Elemente existieren) auch von links- bzw. rechtsinversen Elementen sprechen. Ein Element $ n \in N $ ist linksinvers zu einem gegebenen $ m \in G $ bezüglich eines rechtsneutralen Elements $ e_{G,r } $, falls $ n \boxdot m =e_{G,r} $, wobei wir mit $ e_{G,r}$ ein rechtsneutrales Element bezeichnen. $ n $ wirkt damit auf $ m $ von links ähnlich wie ein inverses Element. 
(Falls $ e_{G,r} $ ein neutrales Element, also zusätzlich linksneutral ist, wirkt $ n $ von links genau wie ein inverses Element auf $m $. 
Man kann aber auch von Linksinversen sprechen, wenn kein neutrales Element existiert, sondern nur ein Rechtsneutrales. )
Umgekehrt ist ein Element $ n \in M $ rechtsinvers (bezüglich eines linksneutralen Elements $ e_{G,l} \in G $, falls $ m \boxdot n= e_{G,l} $, wobei $ e_{G,l} $ ein linksneutrales Element bezeichnet. 

Wenn eine innere Verknüpfung auf einer Menge $ G $ assoziativ ist, so folgen sowohl aus der Existenz eines rechtsneutralen Elements und linksinverser Elemente (bezüglich diesem) als auch aus der Existenz eines linkneutralen Elements und rechtsinverser Elemente (bezüglich diesem) bereits die Gruppeneigenschaften. Existenz rechts- bzw. linksneutraler Elemente meint dabei, dass wir zu jedem Element $ m \in G $ ein rechts- bzw. linksneutrales Element in $ G $ finden.

%%% Local Variables:
%%% mode: LaTeX
%%% TeX-master: "../mfnf"
%%% End:
